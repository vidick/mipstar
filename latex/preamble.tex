%\documentclass{stacks-project-book}
\documentclass{amsart}
\usepackage{amsmath}

% For dealing with references we use the comment environment
\usepackage{verbatim}
\newenvironment{reference}{\comment}{\endcomment}
\newenvironment{slogan}{\comment}{\endcomment}
\newenvironment{history}{\comment}{\endcomment}


% We use multicol for the list of chapters between chapters
\usepackage{multicol}

% This is generall recommended for better output
\usepackage{lmodern}
\usepackage[T1]{fontenc}

% For cross-file-references
\usepackage{xr-hyper}

% Package for hypertext links:
\usepackage{hyperref}

\usepackage{tikz-cd}

% For any local file, say "hello.tex" you want to link to please
% use \externaldocument[hello-]{hello}
\externaldocument[introduction-]{introduction}
\externaldocument[notation-]{notation}
\externaldocument[nonlocalgames-]{nonlocalgames}
\externaldocument[blsgames-]{blsgames}
\externaldocument[complexitytheory-]{complexitytheory}
\externaldocument[operatoralgebras-]{operatoralgebras}
\externaldocument[rigidity-]{rigidity}
\externaldocument[paulibraiding-]{paulibraiding}
\externaldocument[theargument-]{theargument}
\externaldocument[questionreduction-]{questionreduction}
\externaldocument[answerreduction-]{answerreduction}
\externaldocument[recursivecompression-]{recursivecompression}
\externaldocument[classicalldt-]{classicalldt}
\externaldocument[quantumldt-]{quantumldt}
\externaldocument[parallelrepetition-]{parallelrepetition}

% Theorem environments.
%
\theoremstyle{plain}
\newtheorem{theorem}{Theorem}[section]
\newtheorem{proposition}[theorem]{Proposition}
\newtheorem{lemma}[theorem]{Lemma}
\newtheorem{fact}[theorem]{Fact}
\newtheorem{claim}[theorem]{Claim}
\newtheorem{corollary}[theorem]{Corollary}



\theoremstyle{definition}
\newtheorem{definition}[theorem]{Definition}
\newtheorem{example}[theorem]{Example}
\newtheorem{exercise}[theorem]{Exercise}
\newtheorem{situation}[theorem]{Situation}

\theoremstyle{remark}
\newtheorem{remark}[theorem]{Remark}
\newtheorem{remarks}[theorem]{Remarks}

\numberwithin{equation}{subsection}


%%%%%%%%%%
%% Macros

% General

\renewcommand{\eqref}[1]{(\ref{#1})}

\newcommand{\eref}[1]{(\ref{#1})}

% Notation

\newcommand{\cal}[1]{\mathcal{#1}}
\newcommand{\mH}{\mathcal{H}}
\newcommand{\eps}{\varepsilon}

\newcommand{\mA}{\mathcal{A}}
\newcommand{\mB}{\mathcal{B}}
\newcommand{\mG}{\mathcal{G}}
\newcommand{\mI}{\mathcal{I}}
\newcommand{\mP}{\mathcal{P}}
\newcommand{\mR}{\mathcal{R}}
\newcommand{\mS}{\mathcal{S}}
\newcommand{\mX}{\mathcal{X}}
\newcommand{\mY}{\mathcal{Y}}


% Complexity classes
\newcommand{\NP}{\textsc{NP}}
\newcommand{\MIP}{\textsc{MIP}}
\newcommand{\RE}{\textsc{RE}}

% Probability

\newcommand{\Es}[1]{\textsc{E}_{#1}}
\newcommand{\E}{\mathop{\mathbb{E}}\displaylimits} % Expectation


% Math

\newcommand{\C}{\mathbb{C}}
\newcommand{\N}{\mathbb{N}}
\newcommand{\Z}{\mathbb{Z}}




\newcommand{\Tr}{\ensuremath{\mathrm{Tr}}}
\newcommand{\Id}{\ensuremath{\mathop{\rm Id}\nolimits}}

\newcommand{\norm}[1]{\left\| {#1} \right\|}

\newcommand{\eval}{\mathrm{eval}}
\newcommand{\poly}{\mathrm{poly}}



% Quantum notation

\newcommand{\ket}[1]{|#1\rangle}
\newcommand{\bra}[1]{\langle#1|}
\newcommand{\ketbra}[2]{\ket{#1}\!\bra{#2}}
\newcommand{\proj}[1]{\ket{#1}\!\bra{#1}}

\newcommand{\reg}[1]{{\textsf{#1}}}


% Games

\newcommand{\game}{\mathfrak{G}}
\newcommand{\strategy}{\mathcal{S}}
\newcommand{\val}{\ensuremath{\mathrm{val}}}
\newcommand{\Ent}{\mathcal{E}}
\newcommand{\alice}{A}
\newcommand{\bob}{B}

\newcommand{\abc}[1][\delta]{\otimes I_\bob \simeq_{#1} I_\alice \otimes}

\newcommand{\MS}{\mathrm{MS}}
\newcommand{\CHSH}{\mathrm{CHSH}}
