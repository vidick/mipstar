%\documentclass{stacks-project-book}
\documentclass{amsart}
\usepackage{amsmath}

% For dealing with references we use the comment environment
\usepackage{verbatim}
\newenvironment{reference}{\comment}{\endcomment}
\newenvironment{slogan}{\comment}{\endcomment}
\newenvironment{history}{\comment}{\endcomment}


% We use multicol for the list of chapters between chapters
\usepackage{multicol}

% This is generall recommended for better output
\usepackage{lmodern}
\usepackage[T1]{fontenc}

% For cross-file-references
\usepackage{xr-hyper}

% Package for hypertext links:
\usepackage{hyperref}

% For any local file, say "hello.tex" you want to link to please
% use \externaldocument[hello-]{hello}
\externaldocument[introduction-]{introduction}
\externaldocument[-]{}
\externaldocument[nonlocalgames-]{}
\externaldocument[complexitytheory-]{}
\externaldocument[operatoralgebras-]{}

\externaldocument[paulibraiding-]{paulibraiding}


% Theorem environments.
%
\theoremstyle{plain}
\newtheorem{theorem}[subsection]{Theorem}
\newtheorem{proposition}[subsection]{Proposition}
\newtheorem{lemma}[subsection]{Lemma}
\newtheorem{fact}[subsection]{Fact}


\theoremstyle{definition}
\newtheorem{definition}[subsection]{Definition}
\newtheorem{example}[subsection]{Example}
\newtheorem{exercise}[subsection]{Exercise}
\newtheorem{situation}[subsection]{Situation}

\theoremstyle{remark}
\newtheorem{remark}[subsection]{Remark}
\newtheorem{remarks}[subsection]{Remarks}

\numberwithin{equation}{subsection}


%%%%%%%%%%
%% Macros

% Notation

\newcommand{\cal}[1]{\mathcal{#1}}
\newcommand{\mH}{\mathcal{H}}
\newcommand{\eps}{\varepsilon}
\newcommand{\mA}{\mathcal{A}}
\newcommand{\mX}{\mathcal{X}}
\newcommand{\mY}{\mathcal{Y}}
\newcommand{\mR}{\mathcal{R}}
\newcommand{\mG}{\mathcal{G}}


% Complexity classes
\newcommand{\NP}{\textsc{NP}}

% Probability

\newcommand{\Es}[1]{\textsc{E}_{#1}}
\newcommand{\E}{\mathop{\mathbb{E}}\displaylimits} % Expectation


% Math

\newcommand{\C}{\mathbb{C}}
\newcommand{\N}{\mathbb{N}}

\newcommand{\Tr}{\mbox{\rm Tr}}
\newcommand{\Id}{\ensuremath{\mathop{\rm Id}\nolimits}}

\newcommand{\norm}[1]{\left\| {#1} \right\|}

\newcommand{\eval}{\mathrm{eval}}
\newcommand{\poly}{\mathrm{poly}}



% Quantum notation

\newcommand{\ket}[1]{|#1\rangle}
\newcommand{\bra}[1]{\langle#1|}
\newcommand{\ketbra}[2]{\ket{#1}\!\bra{#2}}
\newcommand{\proj}[1]{\ket{#1}\!\bra{#1}}

% Games

\newcommand{\game}{\mathfrak{G}}
\newcommand{\strategy}{{S}}
\newcommand{\val}{\ensuremath{\mathrm{val}}}
\newcommand{\Ent}{\mathcal{E}}
\newcommand{\alice}{A}
\newcommand{\bob}{B}

\newcommand{\abc}[1][\delta]{\otimes I_\bob \simeq_{#1} I_\alice \otimes}


\begin{document}

\part{Background}
\label{book-part-background}


\title{Nonlocal games}
\label{nonlocalgames}

\maketitle

\phantomsection
\label{section-phantom}

\tableofcontents

A \emph{nonlocal game} is another name for a \emph{two-player one-round game}. The term ``nonlocal'' is used to emphasize that players in the game may make use of entanglement to coordinate. However the game itself does not depend on entanglement, and it would be more proper to call the player strategies ``nonlocal''.  

\section{Basic definitions}
\label{section-strat}

\begin{definition}[Two-player one-round games]
  \label{definition-game}
  A \emph{two-player one-round game} $\game$ is a tuple
  $(\cal{X}, \cal{Y}, \cal{A}, \cal{B}, \mu, D)$ where
  \begin{enumerate}
  \item $\cal{X}$ and $\cal{Y}$ are finite sets (called the \emph{question
      alphabets}),
  \item $\cal{A}$ and $\cal{B}$ are finite sets (called the \emph{answer
      alphabets}),
  \item $\mu$ is a probability distribution over $\cal{X} \times \cal{Y}$
    (called the \emph{question distribution}), and
  \item $D: \cal{X} \times \cal{Y} \times \cal{A} \times \cal{B} \to \{0,1\}$ is
    a function (called the \emph{decision predicate}).
  \end{enumerate}
\end{definition}

Given a game, one can consider different kinds of strategies for the players in the game. The most important type of strategy considered here is that of a ``tensor product strategy'', sometimes also referred to loosely as a ``quantum strategy''. (We use the former terminology to distinguish this class of strategies from the more general ``commuting strategies''.)


\begin{definition}[Tensor product strategies]
  \label{definition-tensorstrategy}
  A \emph{tensor product strategy} $\strategy$ for a game $\game = (\cal{X},
  \cal{Y}, \cal{A}, \cal{B}, \mu, D)$ is a tuple $(\rho, A, B)$ where
  \begin{itemize}
	\item $\rho$ is a density matrix in $\mH_A \otimes \mH_B$ for finite
    dimensional complex Hilbert spaces $\mH_A, \mH_B$,
	\item $A$ is a set $\{A^x\}$ such that for every $x \in \cal{X}$, $A^x =
    \{A^x_a \}_{a \in \cal{A}}$ is a POVM over $\mH_A$, and
	\item $B$ is a set $\{B^y\}$ such that for every $y \in \cal{Y}$, $B^y =
    \{B^y_b \}_{b \in \cal{B}}$ is a POVM over $\mH_B$.
\end{itemize}
\end{definition}

Frequently we write a strategy as $(\ket{\psi}, A, B)$, indicating that $\rho = \proj{\psi}$ is taken to be a pure state. This is without loss of generality since any $\rho$ can be purified using an ancilla system that can be added to the space $\mH_A$ or $\mH_B$. 

\begin{definition}[Projective strategies]
  \label{definition-projective-strategy}
  We say that a strategy $\strategy = (\ket{\psi}, A, B)$ is \emph{projective} if
  all the measurements $\{A^x_a\}_a$ and $\{B^y_b\}_b$ are projective.
\end{definition}

\begin{remark}\label{remark-projective}
Using Naimark's theorem one may always, at the cost of extending the spaces $\mH_A$ and $\mH_B$ and tensoring $\ket{\psi}$ with the state $\ket{0}$ on the extension, assume that a strategy is projective. See e.g.~\cite[Theorem 4.2]{NW19}.
\end{remark}

We obtain classical strategies by restricting the state to be separable. 

\begin{definition}[Classical  strategies]
  \label{definition-classicalstrategy}
  A \emph{classical strategy} for a game $\game$ is a {tensor product strategy} $(\rho, A, B)$ such that furthermore the state $\rho$ is separable across $\mH_A$ and $\mH_B$. 
\end{definition}

\begin{remark}
Classical strategies can equivalently be represented as a triple $(\nu,f,g)$ where $\nu$ is a probability measure on $\Omega$ and $f=\{f^x\}$, $g=\{g^y\}$ where $f^x:\Omega\times \cal{X}\to\cal{A}$, $g^y: \Omega\times \cal{Y}\to \cal{B}$ measurable functions. To see this, given $(\rho,A,B)$ where $A^x$ and $B^y$ are projective measurements (see Remark~\ref{remark-projective}) write $\rho = \sum_i p_i \rho^A_i \otimes \rho^B_i$ where $(p_i)_{i\in \cal{I}}$ is a distribution over a finite set $\cal{I}$. Let $\Omega = \cal{I} \times [0,1]$ and $\mu$ the product of $(p_i)$ with the Lebesgue measure on $[0,1]$. Finally, for all $x\in \cal{X}$, order the set $\cal{A}=(a_1,\ldots,a_{|\cal{A}|})$ in an arbitrary way and for $(i,t)\in\cal{I}\times [0,1]$ let $f^x(i,t)$ be the unique $a_k\in \cal{A}$ such that 
\[ \sum_{1\leq j \leq k}  \Tr(A^x_{a_j} \rho_i^A) \leq t < \sum_{1\leq j \leq k+1}  \Tr(A^x_{a_j} \rho_i^A) \;.\]
Proceeding similarly to define $g^y$ one can verify that for all $(a,b)$, 
\[\Tr(A^x_a \otimes B^y_b \rho) = \int 1_{f^x(i,t)=a} 1_{g^y(i,t)=b}d(i,t)\;.\] 
\end{remark}

Commuting strategies extend tensor strategies. 

\begin{definition}[Commuting strategies]
  \label{definition-commutingstrategy}
  A \emph{commuting strategy} $\strategy$ for a game $\game = (\cal{X},
  \cal{Y}, \cal{A}, \cal{B}, \mu, D)$ is a tuple $(\rho, A, B)$ where
  \begin{itemize}
	\item $\rho$ is a density matrix in $\mH$ for separable Hilbert space $\mH$,
	\item $A$ is a set $\{A^x\}$ such that for every $x \in \cal{X}$, $A^x =
    \{A^x_a \}_{a \in \cal{A}}$ is a POVM over $\mH$, 
	\item $B$ is a set $\{B^y\}$ such that for every $y \in \cal{Y}$, $B^y =
    \{B^y_b \}_{b \in \cal{B}}$ is a POVM over $\mH$,
		\item For all $(x,a)$ and $(b,y)$, $A^x_a$ commutes with $B^y_b$. 
\end{itemize}
\end{definition}


Given a game, the most important quantity associated to it is the game $\emph{value}. Informally, this represents the maximum success probability of players employing a certain kind of strategy in the game.  

\begin{definition}[Tensor product value]
  \label{definition-tensor-value}
	The \emph{value} of a tensor product strategy $\strategy =
  (\ket{\psi}, A, B)$  with respect to a game $\game=(\cal{X}, \cal{Y}, \cal{A},
  \cal{B}, \mu, D)$ is defined as
  \begin{equation*}
		\val(\game, \strategy) = \sum_{x,\, y,\, a,\, b} \, \mu(x,y)\, D(x,y,a,b)\,
    \bra{\psi} A^x_a \otimes B^y_b\, \ket{\psi}\;.
  \end{equation*}
	For $v\in[0,1]$ we say that the strategy $\strategy$ \emph{passes (or wins)
    $\game$ with probability $v$ if} $\val^*(\game, \strategy) \geq v$.
  The \emph{tensor product value} of $\game$ is defined as
  \begin{equation*}
		\val^*(\game) = \sup_\strategy \val(\game, \strategy)\;,
  \end{equation*}
	where the supremum is taken over all tensor product strategies $\strategy$ for
  $\game$.
\end{definition}

\begin{definition}[Classical value]
  \label{definition-classical-value}
	The \emph{classical value} of a game $\game$ is defined as
  \begin{equation*}
		\val(\game) = \sup_\strategy \val(\game, \strategy)\;,
  \end{equation*}
	where the supremum is taken over all classical strategies $\strategy = (\rho,A,B)$ for
  $\game$.
\end{definition}


\begin{remark}
  Unless specified otherwise, all strategies considered  are tensor
  product strategies, and we simply call them \emph{strategies}.
  Similarly, we refer to $\val^*(\game)$ as the \emph{value} of the game
  $\game$.
\end{remark}



\begin{definition}
  \label{definition-symmetric-games}
	A game $\game = (\cal{X}, \cal{Y}, \cal{A}, \cal{B}, \mu, D)$ is
  \emph{symmetric} if the question and answer alphabets are the same for both
  players (i.e.\
  $\cal{X} = \cal{Y}$ and $\cal{A} = \cal{B}$), the distribution $\mu$ is
  symmetric (i.e.\
  $\mu(x,y) = \mu(y,x)$), and the decision predicate $D$ treats both players
  symmetrically (i.e.\ for all $x,y,a,b$, $D(x,y,a,b) = D(y,x,b,a)$).
	
  Furthermore, we call a strategy $\strategy = (\ket{\psi}, A, B)$
  \emph{symmetric} if $\ket{\psi}$ is a state in $\mH \otimes \mH$, for some
  Hilbert space $\mH$, that is invariant under permutation of the two factors,
  and the measurement operators of both players are identical.
	
  We specify symmetric games $\game$ and symmetric strategies $\strategy$ using
  a more compact notation: we write $\game = (\cal{X}, \cal{A}, \mu, D)$ and
  $\strategy = (\ket{\psi}, M)$ where $M$ denotes the set of measurement
  operators for both players.
\end{definition}

\begin{lemma}
  \label{lemma-symmetric-strat}
  Let $\game = (\cal{X}, \cal{Y}, \cal{A}, \cal{B}, \mu, D)$ be a symmetric game
  such that $\val^*(\game) = 1-\eps$ for some $\eps\geq 0$.
  Then there exists a symmetric and projective strategy
  $\strategy=(\ket{\psi},M)$ such that $\val^*(\game,\strategy)\geq 1-2\eps$.
\end{lemma}

\begin{proof}
  By definition there exists a strategy $\strategy'= (\ket{\psi'},A,B)$ such
  that $\val^*(\game,\strategy')\geq 1-2\eps$.
Using Naimark's theorem (for example as formulated in~\cite[Theorem 4.2]{NW19}) we can assume
without loss of generality that $\ket{\psi'}\in\C_{\alice'}^d\otimes \C_{\bob'}^d$ for some integer $d$
  and that for every $x$ and $y$, $A^x$ and $B^y$ is a projective measurement.
%  Enlarging one player's space if necessary, assume without loss of generality
%  that $\ket{\psi'}\in\C_{\alice'}^d\otimes \C_{\bob'}^d$ for some integer $d$
%  and that for every $x$ and $y$, $A^x$ and $B^y$ is a projective measurement.
  Let
  \[
    \ket{\psi} = \frac{1}{\sqrt{2}} \big( \ket{0}_\alice \ket{1}_\bob
    \ket{\psi'}_{\alice'\bob'} + \ket{1}_\alice \ket{0}_\bob
    \ket{\psi'_\tau}_{\alice'\bob'}\big) \in (\C_{\alice}^2 \otimes
    \C_{\alice'}^d) \otimes (\C_\bob^2 \otimes \C_{\bob'}^d)\;,
  \]
  where $\ket{\psi'_\tau}$ is obtained from $\ket{\psi}$ by permuting the two
  players' registers.
  Observe that $\ket{\psi}$ is invariant under permutation of $\alice\alice'$
  and $\bob\bob'$.
  
  For any question $x\in \cal{X}=\cal{Y}$, define the measurement $M^x = \{M^x_a\}_{a \in \cal{A}}$
  acting on the Hilbert space $\C^2 \otimes \C^d$ as follows:
  \[
  	M^x_a = \ketbra{0}{0} \otimes A^x_a + \ketbra{1}{1} \otimes B^x_a \;.
  \]
  When Alice receives question $x$, she measures $M^x$ on registers $\alice \alice'$, and when Bob receives 
  question $y$, he measures $M^y$ on registers $\bob \bob'$.
  %let $M^x$ be the measurement obtained
%  by first measuring the qubit in register $w$ and depending on the outcome,
 % applying the measurement $A^x$ or $B^x$ on player $w$'s $d$-dimensional
  %register $w'$ to obtain an outcome $a$.
  Using that by assumption the decision predicate $D$ for $\game$ is symmetric,
  it is not hard to verify that $\val^*(\game,\strategy) =
  \val^*(\game,\strategy')$.
\end{proof}

\begin{definition}
  \label{def:comm-strategy}
  Let $\game = (\cal{X}, \cal{Y}, \cal{A}, \cal{B}, \mu, D)$ be a game, and let
  $\strategy = (\ket{\psi}, A, B)$ be a strategy for $\game$ such that the
  spaces $\mH_A \simeq \mH_B$ canonically.
  Let $S \subseteq \cal{X} \times \cal{Y}$ denote the support of the question
  distribution $\mu$, i.e.\ the set of $(x,y)$ such that $\mu(x,y)>0$.
  We say that $\strategy$ is a \emph{commuting strategy for $\game$} if for all
  question pairs $(x, y) \in S$, we have $[A^x_a, B^y_b] = 0$ 
  for all $a
  \in \cal{A}, b \in \cal{B}$, where $[A, B] = AB - BA$ denotes the commutator.
\end{definition}

\begin{definition}[Consistent measurements]
  \label{def:consistent-measurement}
  Let $\cal{A}$ be a finite set, let $\ket{\psi} \in \mH \otimes \mH$ a
  state, and $\{ M_a \}_{a \in \cal{A}}$ a projective measurement on
  $\mH$.
  We say that \emph{$\{M_a\}_{a \in \cal{A}}$ is consistent on $\ket{\psi}$} if
  and only if
  \begin{equation*}
    \forall a \in \mA\;,\quad M_a \otimes \Id_\bob\, \ket{\psi}
    = \Id_\alice \otimes M_a\, \ket{\psi}\;.
  \end{equation*}
\end{definition}

\begin{definition}[Consistent strategies]
  \label{def:consistent-strategy}
	Let $\strategy = (\ket{\psi}, A, B)$ be a projective strategy with state
  $\ket{\psi} \in \mH \otimes \mH$, for some Hilbert space~$\mH$, which is
  defined on question alphabets $\cal{X}$ and~$\cal{Y}$ and answer alphabets
  $\cal{A}$ and~$\cal{B}$, respectively.
  We say that the strategy $\strategy$ is \emph{consistent} if for all $x \in
  \cal{X}$, the measurement $\{A^x_a\}_{a \in \cal{A}}$ is consistent on
  $\ket{\psi}$ and if for all $y \in \cal{Y}$, the measurement $\{B^y_b\}_{b \in
    \cal{B}}$ is consistent on $\ket{\psi}$.
\end{definition}

\begin{definition}
  \label{def:spcc}
  We say that a strategy $\strategy$ for a game $\game$ is \emph{PCC} if it is projective, consistent, and commuting for $\game$.
  Additionally, we say that a PCC strategy $\strategy$ is \emph{SPCC} if it is furthermore symmetric.
\end{definition}

\begin{remark}
  A strategy $\strategy = (\ket{\psi},A,B)$ for a symmetric game $\game =
  (\cal{X}, \cal{X}, \cal{A}, \cal{A}, \mu, D)$ is called \emph{synchronous} if
  it holds that for every $x\in \cal{X}$ and $a\neq b \in \cal{A}$, $\bra{\psi}
  A^x_a \otimes B^x_b \ket{\psi} = 0$; in other words,
  the players never return different answers when simultaneously asked the same
  question.
  As shown in~\cite{paulsen2016estimating} the condition for a
  finite-dimensional strategy of being synchronous is equivalent to the
  condition that it is projective, consistent, and moreover $\ket{\psi}$ is a
  maximally entangled state.
  (The equivalence is extended to infinite-dimensional strategies, as well as
  correlations induced by limits of finite-dimensional strategies,
  in~\cite{kim2018synchronous}.)
\end{remark}

\begin{definition}[Entanglement requirements of a game]
  \label{def:ent}
	For all games $\game$ and $\nu \in [0, 1]$, let $\Ent(\game, \nu)$ denote the
  minimum integer $d$ such that there exists a finite dimensional tensor product
  strategy $\strategy$ that achieves success probability at least $\nu$ in the
  game $\game$ with a state $\ket{\psi}$ whose Schmidt rank is at most $d$.
  If there is no finite dimensional strategy that achieves success probability
  $\nu$, then define $\Ent(\game, \nu)$ to be $\infty$.
\end{definition}

\section{Distance measures}
\label{section-distancemeasures}

We introduce several distance measures on states and strategies.

\begin{definition}[Distance between states]
  \label{definition-state-distance}

	Let $\{\ket{\psi_n}\}_{n\in \N}$ and $\{\ket{\psi'_n}\}_{n\in \N}$ be two
  families of states in the same space $\mH$.
  For some function $\delta : \N \to [0,1]$ we say that $\{ \ket{\psi_n} \}$ and
  $\{ \ket{\psi'_n} \}$ are \emph{$\delta$-close}, denoted as $\ket{\psi}
  \approx_\delta \ket{\psi'}$, if $ \norm{\ket{\psi_n} - \ket{\psi'_n}}^2 =
  O(\delta(n))$.
  (For convenience we generally leave the dependence of the states and $\delta$
  on the indexing parameter $n$ implicit.)
\end{definition}

\begin{definition}[Consistency between POVMs]
  \label{def:consistency}
	Let $\mX$ be a finite set and $\mu$ a distribution on $\mX$.
  Let $\ket{\psi} \in \cal{H}_A \otimes \cal{H}_B$ be a quantum state, and for
  all $x\in \mX$, $\{A^x_a\}$ and $\{B^x_a\}$ POVMs.
  We write
  \begin{equation*}
    A^x_a \otimes I_\bob \simeq_\delta I_\alice \otimes B^x_a
  \end{equation*}
  on state $\ket{\psi}$ and distribution $\mu$ if
  \begin{equation*}
    \E_{x\sim \mu} \sum_{a\ne b} \bra{\psi} A^x_a \otimes B^x_b \ket{\psi}
    \leq O(\delta)\;.
  \end{equation*}
  In this case, we say that $\{A^x_a\}$ and $\{B^x_a\}$ are
  \emph{$\delta$-consistent} on $\ket{\psi}$.
\end{definition}

Note that a consistent measurement according to
Definition~\ref{def:consistent-measurement} is $0$-consistent with itself, under
the singleton distribution, according to Definition~\ref{def:consistency} (and
vice-versa).

\begin{definition}[Distance between POVMs]
  \label{def:povm-distance}
	Let $\mX$ be a finite set and $\mu$ a distribution on $\mX$. 
  Let $\ket{\psi} \in \cal{H}$ be a quantum state, and for all $x\in \mX$,
  $\{M^x_a\}$ and $\{N^x_a\}$ two POVMs on $\mH$.
  We say that \emph{$\{M^x_a\}$ and $\{N^x_a\}$ are $\delta$-close on state
    $\ket{\psi}$ and under distribution $\mu$} if
  \begin{equation*}
    \E_{x\sim \mu} \sum_a \norm{ (M^x_a - N^x_a) \ket{\psi} }^2 \leq O(\delta)\;,
  \end{equation*}
  and we write $M^x_a \approx_\delta N^x_a$ to denote this when the state
  $\ket{\psi}$ and distribution $\mu$ are clear from context.
  This distance is referred to as the \emph{state-dependent} distance.
\end{definition}

\begin{definition}[Distance between strategies]
  \label{def:strategy-distance}
  Let $\game = (\cal{X}, \cal{Y}, \cal{A}, \cal{B}, \mu, D)$ be a nonlocal game
  and let $\strategy = (\psi, A, B)$, $\strategy' = (\psi', A', B')$ be 
  strategies for $\game$.
  For $\delta\in[0,1]$ we say that \emph{$\strategy$ is $\delta$-close to
    $\strategy'$} if the following conditions hold.
  \begin{enumerate}
	\item The states $\ket{\psi}, \ket{\psi'}$ are states in the same
    Hilbert space $\mH_A \otimes \mH_B$ and are $\delta$-close.
	\item For all $x \in \cal{X}, y \in \cal{Y}$, we have $A^x_a \approx_\delta
    (A')^x_a$ and $B^y_b \approx_\delta (B')^y_b$, with the approximations
    holding under the distribution $\mu$, and on either $\ket{\psi}$ or
    $\ket{\psi'}$.
  \end{enumerate}
\end{definition}

We record several useful facts about the consistency measure and the
state-dependent distance without proof.
Readers are referred to Sections 4.4 and 4.5 in~\cite{NW19} for additional
discussion and proofs.

\begin{fact}[Fact 4.13 and Fact 4.14 in~\cite{NW19}]
  \label{fact:agreement}
  For POVMs $\{A^x_a\}$ and $\{B^x_a\}$, the following hold.
  \begin{enumerate}
  \item If $A^x_a \otimes I_\bob \simeq_\delta I_\alice \otimes B^x_a$ then
    $A^x_a \otimes I_\bob \approx_\delta I_\alice \otimes B^x_a$.
    \label{item:consistency-implies-approx}
  \item If $A^x_a \otimes I_\bob \approx_\delta I_\alice \otimes B^x_a$
    \emph{and} $\{A^x_a\}$ and $\{B^x_a\}$ are projective measurements, then
    $A^x_a \otimes I_\bob \simeq_\delta I_\alice \otimes B^x_a$.
    \label{item:both-projectors-implies-consistency}
  \item If $A^x_a \otimes I_\bob \approx_\delta I_\alice \otimes B^x_a$ and
    either $\{A^x_a\}$ or $\{B^x_a\}$ is a projective measurement, then
    $A^x_a \otimes I_\bob \simeq_{\delta^{1/2}} I_\alice \otimes B^x_a$.
    \label{item:one-projector-implies-consistency}
  \end{enumerate}
\end{fact}

\begin{fact}[Fact 4.20 in~\cite{NW19}]
  \label{fact:add-a-proj}
  Let $\cal{A},\cal{B},\cal{C}$ be finite sets, and let $D$ be a distribution
  over question pairs $(x,y)$.
  Let $\{A_{a,b}^x\}$ and $\{B_{a,b}^x\}$ be operators whose outcomes range over the
  product set $\cal{A} \times \cal{B}$.
  Suppose a set of operators $\{C_{a,c}^y\}$, whose outcomes range over the
  product set $\cal{A} \times \cal{C}$, satisfies the condition $\sum_{a,c}
  (C_{a,c}^y)^\dagger C_{a,c}^y \leq \Id$ for all $y$.
  If $A_{a,b}^x \approx_\delta B_{a,b}^x$ on average over $x$ sampled from the
  corresponding marginal of distribution $D$, then $C^y_{a,c} A^x_{a,b}
  \approx_\delta C^y_{a,c} B^x_{a,b}$ on average over $(x,y)$ sampled from $D$.
\end{fact}
\begin{proof}
  Fix questions $x,y$ and answers $a \in \cal{A},b \in \cal{B}$. We have then that
  \begin{align}
    \sum_{c} \left \| (C^y_{a,c} A^x_{a,b} - C^y_{a,c} B^x_{a,b} ) \ket{\psi} \right \|^2
    & = \sum_{c} \bra{\psi} (A^x_{a,b} - B^x_{a,b})^\dagger (C^y_{a,c})^\dagger (C^y_{a,c})
      (A^x_{a,b} - B^x_{a,b} ) \ket{\psi}\\
    & \leq \bra{\psi} (A^x_{a,b} - B^x_{a,b})^\dagger
      (A^x_{a,b} - B^x_{a,b} ) \ket{\psi} \\
    & = \left \| (A^x_{a,b} - B^x_{a,b})\ket{\psi} \right \|^2
  \end{align}
  where the inequality follows from the fact that $\sum_c (C_{a,c}^y)^\dagger
  C_{a,c}^y \leq \sum_{a,c} (C_{a,c}^y)^\dagger C_{a,c}^y \leq \Id$.
  Thus we obtain the desired conclusion
  \begin{equation*}
    \E_{(x,y) \sim D} \sum_{a,b,c} \left \| (C^y_{a,c} A^x_{a,b} - C^y_{a,c} B^x_{a,b} )
      \ket{\psi} \right \|^2 	\leq \E_{(x,y) \sim D} \sum_{a,b}
    \left \| (A^x_{a,b} - B^x_{a,b})\ket{\psi} \right \|^2 \leq \delta.
  \end{equation*}
\end{proof}


\begin{fact}
\label{fact:add-a-proj2}
Let $\mathcal{X},\mathcal{A}$ denote finite sets, and let $\mathcal{G}$ denote a set of functions $g: \mathcal{X} \to \mathcal{A}$. Let $\{A^x_a \}$, $\{B^x_a\}$ be POVMs indexed by $\mathcal{X}$ and outcomes in $\mathcal{A}$. Let $\{ S_g^x \}$ denote a set of operators such that for all $x \in \mathcal{X}$, $\sum_g (S_g^x)^\dagger S_g^x \leq \Id$. 
If $A^x_a \approx_\delta B^x_a$ on average over $x$, then $S_g^x A^x_{g(x)} \approx_\delta S_g^x B^x_{g(x)}$. 
\end{fact}
\begin{proof}
	We expand:
	\begin{align*}
		\E_x \sum_g \norm{ S^x_g (A^x_{g(x)} - B^x_{g(x)}) \ket{\psi}}^2 &= \E_x \sum_g \bra{\psi} (A^x_{g(x)} - B^x_{g(x)})^\dagger  (S_g^x)^\dagger S_g^x (A^x_{g(x)} - B^x_{g(x)}) \ket{\psi} \\
		&= \E_x \sum_a \bra{\psi} (A^x_a - B^x_a)^\dagger  \Big(\sum_{g: g(x) = a} (S_g^x)^\dagger S_g^x \Big) (A^x_a - B^x_a) \ket{\psi} \\
		&\leq \E_x \sum_a \bra{\psi} (A^x_a - B^x_a)^\dagger (A^x_a - B^x_a) \ket{\psi} \\
		&= \E_x \sum_a \norm{ (A^x_a - B^x_a) \ket{\psi} }^2.
	\end{align*}
	The inequality follows from the fact that $\sum_{g: g(x) = a} (S_g^x)^\dagger S_g^x \leq \sum_g (S_g^x)^\dagger (S_g^x) \leq \Id$. The last line is at most $\delta$ by assumption, and we obtain the desired conclusion.
\end{proof}


\begin{lemma}
\label{lem:cool-closeness-fact}
Let $\mathcal{A}$ be a finite set. Let $\{A^x_a\}_{a \in \mathcal{A}}$ be a projective measurement and let $\{B^x_a\}_{a \in \mathcal{A}}$ be a set of matrices. If $A^x_a \approx_\delta B^x_a$, then for all subsets $S \subseteq \mathcal{A}$, we have
\[
	\sum_{a \in S} A^x_a \approx_\delta \sum_{a \in S} A^x_a \cdot B^x_a.
\]
\end{lemma}
\begin{proof}
We expand:
\begin{align*}
	\E_x \norm{ \sum_{a \in S} \Big(A^x_a - A^x_a \cdot B^x_a\Big) \ket{\psi}}^2 &= \E_x \norm{ \sum_{a \in S} A^x_a \cdot \Big(A^x_a - B^x_a\Big) \ket{\psi}}^2 \\
	&= \E_x \sum_{a \in S} \bra{\psi} (A^x_a - B^x_a)^\dagger A^x_a (A^x_a - B^x_a) \ket{\psi} \\
	&\leq \E_x \sum_a \norm{ (A^x_a -  B^x_a) \ket{\psi}}^2.
\end{align*}
In the second line we used the projectivity of $\{ A^x_a \}$, and in the third line we used that $A^x_a \leq \Id$.
\end{proof}


\begin{fact}[Triangle inequality, Fact 4.28 in~\cite{NW19}]
  \label{fact:triangle}
  If $A_a^x \approx_\delta B_a^x$ and $B_a^x \approx_\epsilon C_a^x$, then
  $A_a^x \approx_{\delta + \epsilon} C_a^x$.
\end{fact}

\begin{fact}[Triangle inequality for ``$\simeq$", Proposition 4.29 in~\cite{ML20}]
 \label{fact:triangle-for-simeq}
 If $A_a^x \otimes I_\bob \simeq_\eps I_\alice \otimes B_a^x$,
 $C_a^x \otimes I_\bob \simeq_\delta I_\alice \otimes  B_a^x$,
 and $C_a^x \otimes I_\bob \simeq_\gamma I_\alice \otimes  D_a^x$,
 then $A_a^x \otimes I_\bob \simeq_{\eps + 2\sqrt{\delta + \gamma}} I_\alice \otimes  D_a^x$.
\end{fact}

\begin{fact}[Data processing, Fact 4.26 in~\cite{NW19}]
  \label{fact:data-processing}
  Suppose $A^x_a \abc B^x_a$.
  Then $A^x_{[f(\cdot) = b]} \abc B^x_{[f(\cdot) = b]}$.
\end{fact}


The state-dependent distance is the right tool for reasoning about the closeness
of measurement operators in a strategy.
The following lemma ensures that, when two families of measurements are close on a
state, changing from one family of measurement to the other only introduces
a small error to the value of the strategy.

\begin{lemma}
  \label{lem:commutation-analysis}
  Let $\{A^x_{a,\, b}\}$, $\{B^x_{a,\, b,\, c}\}$, $\{C^x_{a,\, c}\}$ be POVMs.
  Suppose $\{B^x_{a,\, b,\, c}\}$ is projective, and
  \begin{align*}
    A^x_{a,\, b} \otimes \Id_\bob &\approx_\delta \Id_\alice \otimes B^x_{a,\, b}\;,\\
    C^x_{a,\, c} \otimes \Id_\bob &\approx_\delta \Id_\alice \otimes B^x_{a,\, c}\;.
  \end{align*}
  Then the following approximate commutation relation holds:
  \begin{equation*}
    [A^x_{a,\, b}, C^x_{a,\, c}] \otimes I_\bob \approx_\delta 0\;.
  \end{equation*}
\end{lemma}

\begin{proof}
  Applying Fact~\ref{fact:add-a-proj} to $C^x_{a,\, c} \otimes \Id_\bob
  \approx_\delta \Id_\alice \otimes B^x_{a,\, c}$ and $\{A^x_{a,b} \otimes
  \Id_\bob \}$, we have
  \begin{equation}
    \label{eq:commutation-analysis-1}
    A^x_{a,\, b} C^x_{a,\, c} \otimes \Id_\bob
    \approx_\delta A^x_{a,\, b} \otimes B^x_{a,\, c}\,.
  \end{equation}
  Similarly, applying Fact~\ref{fact:add-a-proj} to $A^x_{a,\, b} \otimes
  \Id_\bob \approx_\delta \Id_\alice \otimes B^x_{a,\, b}$ and $\{\Id_\alice
  \otimes B^x_{a,c}\}$, and using the fact that $\{B^x_{a,b,c}\}$ is projective,
  we have
  \begin{equation}
    \label{eq:commutation-analysis-2}
    \begin{split}
      A^x_{a,\, b} \otimes B^x_{a,\, c}
      & \approx_\delta \Id_\alice \otimes B^x_{a,\, c} B^x_{a,\, b}\\
      & =_{\phantom{\delta}} \Id_\alice \otimes B^x_{a,\, b,\, c}\,.
    \end{split}
  \end{equation}
  Combining Equations~\eref{eq:commutation-analysis-1}
  and~\eref{eq:commutation-analysis-2}, we have
  \begin{equation}
    \label{eq:commutation-analysis-3}
    A^x_{a,\, b} C^x_{a,\, c} \otimes \Id_\bob \approx_\delta
    \Id_\alice \otimes B^x_{a,\, b,\, c}\,.
  \end{equation}
  A similar argument gives
  \begin{equation}
    \label{eq:commutation-analysis-4}
    C^x_{a,\, c} A^x_{a,\, b} \otimes \Id_\bob \approx_\delta
    \Id_\alice \otimes B^x_{a,\, b,\, c}\,.
  \end{equation}
  The claim follows from Equations~\eref{eq:commutation-analysis-3}
  and~\eref{eq:commutation-analysis-4}.
\end{proof}

%The following two lemmas relate the closeness between sets of operators and weighted sums of those operators. 
%\begin{lemma}
%\label{lem:avg-closeness}
%Let $\{A^x \}$ and $\{B^x\}$ be operators indexed by some finite set $\mathcal{X}$, and let $\mu$ be a probability distribution over $\mathcal{X}$. Let $\{\alpha_x \}$ be a set of complex numbers such that $|\alpha_x| \leq 1$. Then if $A^x \approx_\delta B^x$ on average over $x$ drawn from $\mu$, then $\E_x \alpha_x A^x \approx_\delta \E_x \alpha_x B^x$.
%\end{lemma}
%\begin{proof}
%Expand
%\begin{align*}
%	\Big\| \E_x \alpha_x \big ( A^x - B^x \big) \ket{\psi} \Big\|^2 &\leq \Big( \E_x \norm{ (A^x -B^x) \ket{\psi}} \Big)^2 \\
%	&\leq \E_x \norm{(A^x - B^x) \ket{\psi}}^2 \\
%	&\leq \delta\;.
%\end{align*}
%The first inequality is the triangle inequality and the second inequality is Jensen's inequality. 
%\end{proof}
%
%\begin{lemma}
%\label{lem:povm-to-obs}
%Let $\mathcal{A}$ be a finite set. Let $\{A^x_a \}_{a \in \mathcal{A}}$ and $\{B^x_a\}_{a \in \mathcal{A}}$ be matrices and let $\{ \alpha_a \}_{a \in \mathcal{A}}$ be a collection of complex numbers on the unit circle. Let $A^x = \sum_a \alpha_a A^x_a$ and $B^x = \sum_a  \alpha_a B^x_a$. Then 
%\[
%	A^x_a \approx_\delta B^x_a \quad \text{implies} \quad A^x \approx_{\delta'} B^x
%\]
%for $\delta' = |\mathcal{A}| \cdot \delta$.
%\end{lemma}
%\begin{proof}
%Expand
%	\begin{align*}
%		\E_x \norm{(A^x - B^x) \ket{\psi}}^2 &= \E_x \Big\|\sum_a \alpha_a (A^x_a - B^x_a) \ket{\psi}\Big\|^2 \\
%		&\leq \E_x \Big( \sum_a \norm{ (A^x_a - B^x_a) \ket{\psi}} \Big)^2 \\
%		&\leq \E_x |\mathcal{A}| \cdot \sum_a \norm{ (A^x_a - B^x_a) \ket{\psi}}^2 \\
%		&\leq |\mathcal{A}| \cdot \delta\;.
%	\end{align*}
%The first inequality follows from the triangle inequality, the second inequality follows from Cauchy-Schwarz, and the last inequality follows from the assumption that $A^x_a \approx_\delta B^x_a$. 
%\end{proof}
%
%

The following lemma is a slightly modified version of~\cite[Fact~4.34]{NW19}.
\begin{lemma}
  \label{lem:ld-sandwich}
  Let $k \geq 0$ be an integer and let $\eps>0$.
  Let $\mX$ be a finite set and $\mu$ a distribution over $\mX$.
  For each $1 \leq i \leq k$ let $\mG_i$ be a set of functions $g_i : \mY
  \rightarrow \mR_i$ and for each $x \in \mX$ let $\{G^{i,\, x}_g\}_{g\in
    \mG_i}$ be a projective measurement.
  Suppose that for all $i\in \{1,\ldots,k\}$, $\mG_i$ satisfies the following
  property: for any two $g_i \neq g_i' \in \mG_i$, the probability that $g_i(y)
  = g_i'(y)$ over a uniformly random $y \in \mY$ is at most $\eps$.

  Let $\bigl\{ A^{x}_{g_1,\, g_2,\, \ldots\,,\, g_k} \bigr\}$ be a projective
  measurement with outcomes $(g_1,\ldots,g_k) \in \mG_1\times\cdots\times
  \mG_k$.
  For each $1 \leq i \leq k$, suppose that on average over $x \sim \mu$ and
  $y\in \mY$ sampled uniformly at random,
  \begin{equation}
    \label{eq:ld-sandwich-1}
    A^{x}_{[\eval_y(\cdot)_i = a_i]} \abc G^{i,\, x}_{[\eval_y(\cdot)=a_i]}\;.
  \end{equation}
  Define the POVM family $\{C^x_{g_1,\, g_2,\, \ldots\,,\, g_k}\}$, for $x\in \mX$, by
  \begin{equation*}
    C^x_{g_1,\, g_2,\, \ldots\,,\, g_k} = G^{k,\, x}_{g_k} \cdots
    G^{2,\, x}_{g_2} \, G^{1,\, x}_{g_1} \, G^{2,\, x}_{g_2} \cdots
    G^{k,\, x}_{g_k}\;.
  \end{equation*}
  Then on average over $x\sim \mu$ and $y\in \mY$ sampled uniformly at random,
  \begin{equation}
    \label{eq:ld-sandwich-2}
    A^x_{[\eval_y(\cdot) = (a_1,\, a_2,\, \ldots,\, a_k)]} \abc[k(\delta+\eps)^{1/2}]
    C^x_{[\eval_y(\cdot) = (a_1,\, a_2,\, \ldots,\, a_k)]}\;.
\end{equation}
\end{lemma}

\begin{proof}
  The proof is identical to the one given in \cite[Fact~4.34]{NW19}, with the
  only modification needed to insert the dependence on $x$ for all measurements
  considered.
\end{proof}


\begin{lemma}[Fact 4.35 in~\cite{NW19}]
  \label{lem:pasting}
Let $\mathcal{D}$ be a distribution on $(x,y_1,y_2)\in \mathcal{X}\times \mathcal{Y}_1\times \mathcal{Y}_2$. For $i\in \{1,2\}$ let $\mathcal{G}_i$ be a collection of functions $g_i: \mathcal{Y}_i \to \mathcal{R}_i$ and let $\{(G_i)^{x}_{g}\}_{g\in \mathcal{G}_i}$ be families of measurements such that $\{(G_2)^x_g\}_g$ is projective for every $x$. Suppose further that for every $(x,y_1)$ it holds that for $g_2\neq g'_2 \in \mathcal{G}_2$ the probability, on average over $y_2$ chosen from $\mathcal{D}$ conditioned on $(x,y_1)$, that $g_2(y_2)=g'_2(y_2)$  is at most $\eta$. 
Let $\{A^{x,y_1,y_2}_{a_1,a_2}\}$ be a family of projective measurements with outcomes $(a_1,a_2)\in \mathcal{R}_1 \times \mathcal{R}_2$ such that for $i\in \{1,2\}$,
\begin{equation}
  \label{eq:pasting-1}
 A^{x,y_1,y_2}_{a_i} \otimes \Id \simeq_\delta \Id \otimes (G_i)^x_{[\eval_{y_i}(\cdot)=a_i]}
\end{equation}
and 
\begin{equation}
  \label{eq:pasting-2}
 A^{x,y_1,y_2}_{a_1,a_2} \otimes \Id \simeq_\delta \Id \otimes  A^{x,y_1,y_2}_{a_1,a_2}\;.
\end{equation}
Define a family of measurements $\{J^x_{g_1,g_2}\}$ as
\begin{equation}
  \label{eq:pasting-2a}
 J^x_{g_1,g_2} \,=\, (G_2)^x_{g_2} (G_1)^x_{g_1} (G_2)^x_{g_2}\;.
\end{equation}
Then there is a 
\begin{equation}
  \label{eq:def-deltap}
\delta_{pasting} \,=\, \delta_{pasting}(\eta,\delta)\,=\, \poly(\eta,\delta)
\end{equation} such that 
\begin{equation}
  \label{eq:pasting-3}
 A^{x,y_1,y_2}_{a_1,a_2} \otimes \Id \simeq_{\delta_p} \Id \otimes J^x_{[\eval_{y_1}(\cdot)=a_1,\eval_{y_2}(\cdot) = a_2]}\;.
\end{equation}
\end{lemma}

\begin{multicols}{2}[\section{Chapters}]
\noindent
Preliminaries
\begin{enumerate}
\item \hyperref[introduction-section-phantom]{Introduction}
\item \hyperref[notation-section-phantom]{Notation}
\end{enumerate}
Background
\begin{enumerate}
\setcounter{enumi}{2}
\item \hyperref[nonlocalgames-section-phantom]{Nonlocal games}
\item \hyperref[complexitytheory-section-phantom]{Complexity theory}
\item \hyperref[operatoralgebras-section-phantom]{Operator algebras}
\end{enumerate}
Warm-up
\begin{enumerate}
\setcounter{enumi}{5}
\item \hyperref[rigidity-section-phantom]{Rigidity}
\item \hyperref[blsgames-section-phantom]{Binary Linear System games}
\item \hyperref[paulibraiding-section-phantom]{Pauli braiding}
\end{enumerate}
Overview
\begin{enumerate}
\setcounter{enumi}{8}
\item \hyperref[argument-section-phantom]{The argument}
\item \hyperref[questionreduction-section-phantom]{Question reduction}
\item \hyperref[answerreduction-section-phantom]{Answer reduction}
\item \hyperref[recursivecompression-section-phantom]{Recursive compression}
\end{enumerate}
Building blocks
\begin{enumerate}
\setcounter{enumi}{12}
\item \hyperref[classicalldt-section-phantom]{Classical low-degree test}
\item \hyperref[quantumldt-section-phantom]{Quantum low-degree test}
\end{enumerate}
Related tools
\begin{enumerate}
\setcounter{enumi}{14}
\item \hyperref[parallelrepetition-section-phantom]{Parallel repetition}
\end{enumerate}
Extensions
\begin{enumerate}
\setcounter{enumi}{15}
\item TBD
\end{enumerate}
\end{multicols}

\bibliography{my}
\bibliographystyle{amsalpha}

\end{document}