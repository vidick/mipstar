%\documentclass{stacks-project-book}
\documentclass{amsart}
\usepackage{amsmath}

% For dealing with references we use the comment environment
\usepackage{verbatim}
\newenvironment{reference}{\comment}{\endcomment}
\newenvironment{slogan}{\comment}{\endcomment}
\newenvironment{history}{\comment}{\endcomment}


% We use multicol for the list of chapters between chapters
\usepackage{multicol}

% This is generall recommended for better output
\usepackage{lmodern}
\usepackage[T1]{fontenc}

% For cross-file-references
\usepackage{xr-hyper}

% Package for hypertext links:
\usepackage{hyperref}

% For any local file, say "hello.tex" you want to link to please
% use \externaldocument[hello-]{hello}
\externaldocument[introduction-]{introduction}
\externaldocument[-]{}
\externaldocument[nonlocalgames-]{}
\externaldocument[complexitytheory-]{}
\externaldocument[operatoralgebras-]{}

\externaldocument[paulibraiding-]{paulibraiding}


% Theorem environments.
%
\theoremstyle{plain}
\newtheorem{theorem}[subsection]{Theorem}
\newtheorem{proposition}[subsection]{Proposition}
\newtheorem{lemma}[subsection]{Lemma}
\newtheorem{fact}[subsection]{Fact}


\theoremstyle{definition}
\newtheorem{definition}[subsection]{Definition}
\newtheorem{example}[subsection]{Example}
\newtheorem{exercise}[subsection]{Exercise}
\newtheorem{situation}[subsection]{Situation}

\theoremstyle{remark}
\newtheorem{remark}[subsection]{Remark}
\newtheorem{remarks}[subsection]{Remarks}

\numberwithin{equation}{subsection}


%%%%%%%%%%
%% Macros

% Notation

\newcommand{\cal}[1]{\mathcal{#1}}
\newcommand{\mH}{\mathcal{H}}
\newcommand{\eps}{\varepsilon}
\newcommand{\mA}{\mathcal{A}}
\newcommand{\mX}{\mathcal{X}}
\newcommand{\mY}{\mathcal{Y}}
\newcommand{\mR}{\mathcal{R}}
\newcommand{\mG}{\mathcal{G}}


% Complexity classes
\newcommand{\NP}{\textsc{NP}}

% Probability

\newcommand{\Es}[1]{\textsc{E}_{#1}}
\newcommand{\E}{\mathop{\mathbb{E}}\displaylimits} % Expectation


% Math

\newcommand{\C}{\mathbb{C}}
\newcommand{\N}{\mathbb{N}}

\newcommand{\Tr}{\mbox{\rm Tr}}
\newcommand{\Id}{\ensuremath{\mathop{\rm Id}\nolimits}}

\newcommand{\norm}[1]{\left\| {#1} \right\|}

\newcommand{\eval}{\mathrm{eval}}
\newcommand{\poly}{\mathrm{poly}}



% Quantum notation

\newcommand{\ket}[1]{|#1\rangle}
\newcommand{\bra}[1]{\langle#1|}
\newcommand{\ketbra}[2]{\ket{#1}\!\bra{#2}}
\newcommand{\proj}[1]{\ket{#1}\!\bra{#1}}

% Games

\newcommand{\game}{\mathfrak{G}}
\newcommand{\strategy}{{S}}
\newcommand{\val}{\ensuremath{\mathrm{val}}}
\newcommand{\Ent}{\mathcal{E}}
\newcommand{\alice}{A}
\newcommand{\bob}{B}

\newcommand{\abc}[1][\delta]{\otimes I_\bob \simeq_{#1} I_\alice \otimes}


\begin{document}

\part{Warm-up}
\label{book-part-warmup}

\title{Rigidity}
\label{rigidity}

\maketitle

\phantomsection
\label{section-phantom}

\tableofcontents

We give an introduction to the phenomenon of rigidity of a nonlocal game. Informally, this is the idea that for some nonlocal games $\game$ whenever a tensor product strategy $\strategy = (\ket{\psi},A,B)$ for $\game$ succeeds with probability $\val^*(G)$ or close to it the strategy must have a specific form; in particular, the state $\ket{\psi}$ may be required to be equivalent to a specific state under local unitaries and the operators $A^x_a$, $B^y_b$ may be required to satisfy certain algebraic relations. 

\paragraph{Resources.} \label{section-resources} Cleve lecture notes. UCSD notes. 

Before getting into the general theory we start with a concrete example: the CHSH game. 

\section{The CHSH game}
\label{section-chsh}

\begin{definition}
\label{definition-chsh}
The CHSH game is the game $\game_\CHSH$ such that $\mX = \mY = \mA = \mB = \{0,1\}$, $\mu$ is uniform on $\mX \times \mY$, and $D(x,y,a,b) = 1$ if $a\oplus b = x\wedge y$ and $0$ otherwise. 
\end{definition}

The CHSH game is first considered in the Physics literature as a Bell inequality~\cite{clauser1969proposed}. It appears as a nonlocal game in~\cite{cleve2004consequences}. Its central role in the study of either type of object is justified by the fact that it is the simplest game, where simplicity is measured by question and answer alphabet sizes, whose classical and quantum values are separated. 

\begin{lemma}\label{lemma-chsh-qval}
The classical value $\val(\game_\CHSH)=\frac{3}{4}$. The tensor product and commuting values $\val^*(\game_\CHSH)=\val^c(\game_\CHSH)= \frac{1}{2} + \frac{1}{2\sqrt{2}}$. The non-signaling value $\val^{ns}(\CHSH)=1$. 
\end{lemma}

\begin{proof}
A strategy such that both players always return $0$ achieves value $\frac{3}{4}$. It is not hard to verify by enumeration that this is optimal. For a more instructive proof, fix a classical deterministic strategy $\strategy$ represented by $f,g:\{0,1\}\to\{0,1\}$. For $x,y\in\{0,1\}$ let $a_x = (-1)^{f(x)}$ and $b_y = (-1)^{g(y)}$. Then 
\begin{align}
\val(\game_\CHSH;\strategy) &= \Pr_{(x,y)\sim \mu} \big( f(x)\oplus g(y) = x\wedge y \big) \label{equation-chsh-qval1}\\
 &= \Pr_{(x,y)\sim \mu} \big( a_x b_y = (-1)^{x\wedge y} \big) \notag\\
&= \frac{1}{4}\sum_{x,y\in\{0,1\}}\frac{1}{2}\big(1 + (-1)^{x\wedge y} a_x b_y\big)\notag\\
&= \frac{1}{2} + \frac{1}{8}\big( a_0 b_0 + a_1 b_0 + a_0 b_1 - a_1 b_1 \big)\notag\\
 &= \frac{1}{2} + \frac{1}{8}\big( a_0 (b_0 + b_1) + a_1 (b_0 -  b_1) \big)\notag\\
&\leq \frac{1}{2} + \frac{1}{8} \cdot 2\,=\, \frac{3}{4}\;.\notag
\end{align}
Here the key step is the last step, where we used that for $b_0,b_1 \in \{\pm 1\}$ exactly one of $b_0+b_1$ and $b_0-b_1$ must be $0$, and the other is $\pm 2$. 

This shows that $\val(\game_\CHSH)=\frac{3}{4}$. We now extend the argument to the quantum case. First we let the reader verify that if $\ket{\psi} = \frac{1}{\sqrt{2}} (\ket{00} + \ket{11})$ is an EPR pair, $A^0, A^1$ measurements in the eigenbases of $\sigma_Z,\sigma_X$ respectively and $B^0,B^1$ measurements in the eigenbases of $H$ and $\sigma_Z H \sigma_Z$ then the strategy $(\ket{\psi},\{A^0,A^1\},\{B^0,B_1\})$ achieves a success probability of $\cos^2 \frac{\pi}{8}$ in the game. To show a matching upper bound, fix any projective strategy $(\ket{\psi}, A^x,B^y\}$. Letting $A_x = A^x_0-A^x_1$ and  $B^y = B^y_0-B^y_1$ an analogous derivation to~\ref{rigidity-equation-chsh-qval1} gives 
\begin{align*}
\val(\game_\CHSH;\strategy) &=\frac{1}{2} + \frac{1}{8}\bra{\psi} A_0 B_0 + A_0 B_1 + A_1 B_0 - A_1 B_1 \ket{\psi}\;.
\end{align*}
The idea, due to Tsirelson~\cite{cirel1980quantum}, is to evaluate the square
\begin{align}
(A_0 B_0 + A_0 B_1 + A_1 B_0 &- A_1 B_1 )^2\label{equation-bellop}\\
 &=\big( (A_0+A_1) B_0 + (A_0-A_1) B_1 \big)^2\notag\\
&= 4\Id\otimes \Id + (A_1A_0-A_0A_1) (B_0B_1-B_1B_0)\;.\notag
\end{align}
Here the second line follows by expanding all squares, using that each operator squares to identity, and $A$ and $B$ operators commute. The last term above has norm at most $8$, whcih shows that $A_0 B_0 + A_0 B_1 + A_1 B_0 - A_1 B_1 $ has norm at most $\sqrt{8}$; the claimed bound on the commuting value of the CHSH game  follows. 
\begin{equation}\label{equation-bellop-2}
x+1=2
\end{equation}
Reference~\ref{rigidity-equation-bellop-2}.

Finally, for the non-signaling value we observe that the family of distributions that for each pair of questions is uniform over all accepting answers is non-signaling. 
\end{proof}


The proof of Lemma~\ref{rigidity-lemma-chsh-qval} identifies a specific quantum tensor product strategy which achieves a value of $\frac{1}{2} + \frac{1}{2\sqrt{2}}$ in the CHSH game. The phenomenon of \emph{rigidity} for a nonlocal game is that optimal, or even near-optimal, strategies are virtually unique. In Lemma~\ref{rigidity-lemm-rigid-chsh-exact}, generalized in Theorem~\ref{rigidity-theorem-rigid-chsh} we show this for the CHSH game. Before showing the lemma we establish a simple geometric lemma, that we greatly generalize later in Theorem~\ref{theorem-gh}. 

\begin{lemma}\label{lemma-ac-exact}
Let $\rho$ be a density matrix on a separable Hilbert space $\mH$. Let $X,Z$ be binary observables on $\mH$ such that $\Tr( (XZ+ZX)^2 \rho)=0$. Then there exists an isometry $V: \mH \to \C^2 \otimes \mH'$ such that for $W\in\{X,Z\}$,
\[ VW\rho W^\dagger V^\dagger \,=\, (\sigma_X \otimes \Id)V\rho V^\dagger (\sigma_X \otimes \Id)^\dagger\;.\]
\end{lemma}

\begin{proof}
We first show the lemma in the case $\rho$ is invertible. The proof is based on Jordan's lemma (also known as the ``CS decomposition'', which we restate as follows. 


\begin{lemma}\label{lemma-cs-dec}
Let $P,Q$ be projections on a separable Hilbert space $\mH$. Then there exists an orthogonal decomposition $\mH = \oplus_i \mS_i$ such that each $\mS_i$ is a $1$- or $2$-dimensional subspace that is stable by $P$ and $Q$. Furthermore, whenever $\mS_i$ is $2$-dimensional there is a basis for it in which $P$ and $Q$ take the form
\begin{equation}\label{eq:pq-form}
 P = \begin{pmatrix} 1 & 0 \\ 0 & 0 \end{pmatrix} \qquad \text{and} \qquad Q = \begin{pmatrix} c^2 & cs \\ cs & s^2 \end{pmatrix}\;,
\end{equation}
 for some $c_i$ and $s_i$ that may depend on $\mS_i$.\footnote{For the case of $1$-dimensional subspaces, since $P$ and $Q$ are projections they are each either identically $0$ or identity in those subspaces.}
\end{lemma} 

There are many possible proofs for Lemma~\ref{lemma-cs-dec}. One possibility is to diagonalize the operator $R=P+Q$ and observe that eigenvectors come in pairs. See e.g.~\url{https://cims.nyu.edu/~regev/teaching/quantum_fall_2005/ln/qma.pdf}. 

Applying the lemma to $P = \frac{1}{2}(Z+\Id)$ and $Q = \frac{1}{2}(X+\Id)$ yields a decomposition $\mH = \oplus_i \mS_i$. Using $XZ=-ZX$ it follows that there are no $1$-dimensional subspaces. In each two-dimensional subspace, necessarily $\Tr(X)=\Tr(Z)=0$ since otherwise one of them is a multiple of identity and commutes with the other. It follows that there exist a choice of basis $(\ket{e_{1,i}},\ket{e_{2,i}})$ for $\mS_i$ such that $Z = \sigma_Z$, and necessarily $X = u \sigma_X$ for some $u\in \{\pm 1\}$. If $u=-1$ then $\ket{e_{2,i}} \to - \ket{e_{2,i}}$ brings both operators to the desired form. 

In case $\rho$ is not invertible we note that for each $1$-dimensional subspace $\mS_i$ it must be that $\Pi_i \rho \Pi_i$ where $\Pi_i$ is the projection on $\mS_i$, and for each two-dimensional subspace either $XZ+ZX=0$ or $\Pi_i \rho \Pi_i$ has full support, because $(XZ+ZX)^2$ is always a multiple of the identity. The isometry $U$ embeds in $\oplus \mS'_i$ where $\mS'_i$ is $\mS_i$ if $\mS_i$ is two dimensional, and $\mS_i \oplus \mS_i$ in case it is one-dimensional.
\end{proof}







\begin{lemma}\label{lemma-rigid-chsh-exact}
Let $\strategy = (\ket{\psi},A^x,B^y)$ be a projective tensor product strategy such that $\val(\game;\strategy)=\val^*(\game_\CHSH;\strategy)$. Then there exists isometries $V_A:\mH_A \to \C^2 \otimes \mH'_A$ and $V_A:\mH_B \to \C^2 \otimes \mH'_B$ such that 
\begin{equation}\label{equation-chsh-state-0}
 V_A \otimes V_B \ket{\psi} \,=\, \frac{1}{\sqrt{2}}\big(\ket{00}+\ket{11}\big) \otimes \ket{\psi'}\;,
\end{equation}
for some state $\ket{\psi'} \in \mH'_A \otimes \mH'_B$,
and 
\begin{equation}\label{eq:chsh-alice}
( V_A \otimes V_B)( A_0 \otimes \Id )\ket{\psi} \,=\, (\sigma_Z \otimes \Id)\frac{1}{\sqrt{2}}(\ket{00}+\ket{11}) \otimes \ket{\psi'} \big\| \;,
\end{equation}
and a similar relation holds with $A_0$ replaced by $A_1$ and $\sigma_Z$ replaced by $\sigma_X$. Moreover, analogous relations hold for Bob's observables. 
\end{lemma}


\begin{proof}
Fix a  projective tensor product strategy $\strategy = (\ket{\psi},A^x,B^y)$ such that $\val(\game;\strategy)=\val^{com}(\game_\CHSH;\strategy)$. Let $A_x = A^x_0-A^x_1$ and $B_y = B^y_0-B^y_1$ be observables associated with this strategy.

Keeping in mind the proof of Lemma~\ref{lemma-chsh-qval}, the strategy is optimal if and only if 
\[\bra{\psi}(A_1A_0-A_0A_1) (B_0B_1-B_1B_0)\ket{\psi} \,=\, 4\;,\]
which immediately implies that $\|(A_1A_0-A_0A_1) \ket{\psi}\|=2$ hence $A_0A_1 \ket{\psi} = - A_1 A_0 \ket{\psi}$. The conclusion follows by applying Lemma~\ref{lemma-ac-exact} to the reduced density $\rho_A$ of $\ket{\psi}$ on $\mH_A$ together with $A_0$ and $A_1$. This yields the isometry $V_A$. A similar argument applies to the observables $B_0$ and $B_1$. 
\end{proof}

\begin{remark} Discuss the commuting case.
\end{remark}



Later in the chapter we show the following extension of Lemma~\ref{lemma-rigid-chsh-exact} to the case of near-optimal strategies: 


\begin{theorem}\label{theorem-rigid-chsh}
Let $\eps>0$, and suppose that a strategy for the players  in the CHSH game, using  a bipartite state $\ket{\psi}\in\C^d\otimes \C^d$ and observables $A_0,A_1$ for Alice and $B_0,B_1$ for Bob, achieves a bias at least $\sqrt{2}/2-\eps$ in the game. Then there are local isometries $V_A,V_B:\C^d \to \C^2 \otimes \C^{d'}$ such that 
\begin{equation}\label{eq:chsh-state}
\big\| V_A \otimes V_B \ket{\psi} - \frac{1}{\sqrt{2}}\big(\ket{00}+\ket{11}\big) \otimes \ket{\psi'} \big\|^2 = O(\sqrt{\eps}) \;,
\end{equation}
and 
\begin{equation}\label{eq:chsh-alice}
\big\|( V_A \otimes V_B)( A_0 \otimes \Id )\ket{\psi} - (\sigma_X \otimes \Id)\frac{1}{\sqrt{2}}(\ket{00}+\ket{11}) \otimes \ket{\psi'} \big\| \,=\, O\big(\sqrt{\eps}\big)\;,
\end{equation}
and a similar relation holds with $A_0$ replaced by $A_1$ and $\sigma_X$ replaced by $\sigma_Z$. Moreover, analogous relations hold for Bob's observables. 
\end{theorem}


Furthermore, under the assumption that the strategy achieves a bias of at least $\sqrt{2}/2-\eps$ in the game, using $|\bra{\psi}X\ket{\psi}|^2 \leq \bra{\psi}XX^*\ket{\psi}$ for any Hermitian $X$, the left-hand side of~\eqref{eq:bellop}, when evaluated on $\ket{\psi}$, must be at least $8(1-\eps)^2$. Applying the Cauchy-Schwarz inequality it follows that both conditions
$$\Tr((A_1A_0 + A_0A_1)^2\rho_A) = O({\eps})\qquad \text{and}\qquad \Tr((B_0B_1+B_1B_1)^2 \rho_B) = O({\eps})$$
must hold, where $\rho_A$ and $\rho_B$ are the reduced density matrices of $\ket{\psi}$ on Alice and Bob respectively. 
Using the notation introduced in Section~\ref{sec:approx-group}, this says that 
\begin{equation}\label{eq:a-ac}
\|A_0A_1 + A_1A_0\|_{\rho_A}^2 = O({\eps})\;,
\end{equation}
 i.e. $A_0$ and $A_1$ approximately commute. 

Now here comes the key observation. Consider the $8$-element group $H$ generated by the Pauli matrices $\sigma_X$ and $\sigma_Z$, i.e. 
$$H\,=\,\pm\big\{ \Id,\, \sigma_X,\,\sigma_Z,\, \sigma_X\sigma_Z\big\}\;.$$
Then I claim that $A_0$ and $A_1$ induce an approximate representation of $H$, by setting 
$$ f(\pm \Id)=\pm\Id,\quad f(\pm\sigma_X)=\pm A_0,\quad\quad f(\pm\sigma_Z)=\pm A_1,\quad f(\pm\sigma_X\sigma_Z) = \pm A_0A_1\;.$$
Note that this is a legal definition, since $A_0$, $A_1$, and $A_0A_1$ are all unitary. Moreover, using only~\eqref{eq:a-ac} and the fact that $A_0$ and $A_1$ are observables, it is immediate to verify that the conditions of Theorem~\ref{thm:gh} are satisfied, i.e. $f$ is an $(O({\eps}),\rho_A)$-representation of $H$. 

[Show approximate rigidity directly --- cf. UCSD notes, and Cleve notes]

\section{Stability}
\label{section-stability}

Our approach for showing rigidity of the CHSH game proceeds in three steps. First, we use the game winning condition to deduce algebraic relations on operators derived from the players' strategy. These relations are approximate and the error is measured in a norm that is reminiscent of a dimension-normalized Frobenius norm, weighted by the state that is part of the strategy. Second, we observe that operators that exactly satisfy the relations have a particularly nice form, and moreofer operators that approximately satisfy them must be close to operators having the nice form. Finally, we use the form obtained for the operators to derive conditions on the state. 

The second part of this strategy can be interpreted as a \emph{stability} result for representations of the group generated by the Pauli $\sigma_X$ and $\sigma_Z$ matrices, for a specific notion of distance. In this section we investigate this notion of stability and prove a result of Gowers and Hatami which shows that all finite groups are stable under this notion. We state the results in general, even though for our purposes the only group to which we will apply it is the Weyl-Heisenberg group generated by $\sigma_X$ and $\sigma_Z$, which will give rigidity for the CHSH and Magic Square (Section~\ref{test}) games, and $n$-fold tensor products of it, which will give rigidity for the Pauli basis test (Chapter~\ref{paulibraiding}).


\subsection{Approximate group representations}
\label{section-approxrep}

For $d$-dimensional matrices  $A,B$ and $\sigma$ such that $\sigma$ is positive semidefinite, write 
$$\langle A,B\rangle_\sigma = \mathrm{Tr}(AB^\dagger \sigma)\;,$$
where as usual we use $B^\dagger$ to denote the conjugate-transpose. Note that the matrix trace inner product is recovered for $\sigma = \Id$. If $\sigma$ is the totally mixed state, then we obtain a dimension-normalized variant of the trace inner product. We will also write $\|A\|_\sigma = \langle A,A\rangle_\sigma^{1/2}$. 

Given an arbitrary finite group $G$ (not necessarily abelian), a group representation of $G$ is a map $f:G \to U_d(\C)$, the group of $d\times d$ unitary matrices, such that $f$ is a homomorphism: for any $x,y\in G$, $f(x^{-1}y)=f(x)^* f(y)$, where we used $^*$ to denote the conjugate transpose (which, for unitary matrices, corresponds to taking the inverse). The following definition introduces a notion of \emph{approximate} group representation.  

\begin{definition}
\label{def:approx-rep}
Given a finite group $G$, an integer $d\geq 1$, $\eps\geq 0$, and a $d$-dimensional positive semidefinite matrix $\sigma$ with trace $1$, an $(\eps,\sigma)$-representation of $G$ is a function $f: G \to U_d(\C)$, the unitary group of $d\times d$ matrices, such that 
\begin{equation}
\label{eq:gh-condition}
\Es{x,y\in G} \,\Re\big(\big\langle f(x)^\dagger f(y) ,f(x^{-1}y) \big\rangle_\sigma\big) \,\geq\, 1-\eps\;,
\end{equation} 
where the expectation is taken under the uniform distribution over $G$.
\end{definition}

Note that the condition~\eref{eq:gh-condition} is equivalent to 
\begin{equation}
\label{eq:gh-condition-2}
\Es{x,y\in G} \, \big\| f(x^{-1}y) - f(x)^\dagger f(y) \big\|_\sigma^2 \,\leq\, 2\eps\;.
\end{equation}
Taking $\eps=0$ and $\sigma$ any invertible positive definite matrix, we see that the case $\eps=0$ corresponds to an \emph{exact} representation of $G$. 

\begin{example}
\label{ex:wh-1}
Consider the Weyl-Heisenberg group $\mP$, which is the group generated by the Pauli $\sigma_X$ and $\sigma_Z$ matrices. It is not hard to verify that this group has $8$ elements, which can be decomposed as $(-1)^c \sigma_X^a \sigma_Z^b$ for $a,b,c\in\{0,1\}$. Any pair of observables $(X,Z)$ such that $\{X,Z\}=0$ can be used to specify a $(0,\sigma)$ representation of $\mP$ for \emph{any} $\sigma$ as follows:
\begin{equation}
\label{eq:def-rep-pauli}
 f\big((-1)^c \sigma_X^a \sigma_Z^b\big) \,=\, (-1)^c X^a Z^b\;,
\end{equation}
for all $a,b,c\in\{0,1\}$. It is immediate to verify that for all $(x,y)\in \mP$ we have $f(x)^*f(y)=f(x^{-1}y)$ and so~\eref{eq:gh-condition} holds with $\eps=0$ for any $\sigma$.
\end{example}

The following simple and recommended exercise asks you to generalize the example to the approximate case. 

\begin{exercise}
\label{ex:wh-2}
Let $(\ket{\psi},X,Z)$ be such that $\|\{X,Z\}\ket{\psi}\|\leq \eps$. Define $f:\mP \mapsto U(\mH)$ as in~\eref{eq:def-rep-pauli}. Show that $f$ is an $(O(\sqrt{\eps}),\sigma)$-representation of $\mP$. 
\end{exercise}

\begin{remark}
The condition \eref{eq:gh-condition} in Definition~\ref{def:approx-rep} is very closely related to Gowers' $U^2$ norm
$$\|f\|_{U^2}^4 \,=\, \Es{xy^{-1}=zw^{-1}}\, \big\langle f(x)f(y)^* ,f(z)f(w)^* \big\rangle_\sigma.$$
While a large Gowers norm implies closeness to an affine function, we are interested in testing homomorphisms, and the condition \eref{eq:gh-condition} will arise naturally from our calculations in the next section. 
\end{remark}

\subsection{The Gowers-Hatami theorem}

There are many possible notions of approximate group representation. Traditionally the most frequently considered one replaces the norm in Definition~\ref{def:approx-rep} by the operator norm. An inconvenience of that variant is that in general approximate representations are not always close to exact representations (see, for example, the famous problem on ``approximately commuting'' versus ``nearly commuting'' operators). In contrast 
Gowers and Hatami~\cite{gowers2017inverse} showed that in the case of Definition~\ref{def:approx-rep}, approximate group representations can always be ``rounded'' to a nearby exact representation.\footnote{We are barely scratching the surface of a growing theory of ``stability'' for group homomorphisms; see e.g.~\cite{becker2020stability} for an introduction and discussion of related notions.} 
We state and prove a slightly more general, but quantitatively weaker, variant of the Gowers-Hatami result.

\begin{theorem}[Gowers-Hatami]
\label{thm:gh}
Let $G$ be a finite group, $\eps\geq 0$, and $f:G\to U_d(\C)$ an $(\eps,\sigma)$-representation of $G$. Then there exists a $d'\geq d$, an isometry $V:\C^d\to \C^{d'}$, and a representation $g:G\to U_{d'}(\C)$ such that 
$$\Es{x\in G}\, \big\| f(x) - V^*g(x)V \big\|_\sigma^2\, \leq\, 2\,\eps.$$ 
\end{theorem}

Gowers and Hatami limit themselves to the case of $\sigma = d^{-1}I_d$, which corresponds to the dimension-normalized Frobenius norm. In this scenario they in addition obtain a tight control of the dimension $d'$, and show that one can always take $d'\ = (1+O(\eps))d$ in the theorem. We will see a much shorter proof than theirs (the proof is implicit in their argument) that does not seem to allow to recover this estimate. The extension to general $\sigma$, however, will be necessary for our purposes.

Note that  Theorem \ref{thm:gh} does not in general hold  with $d'=d$. The reason is that it is possible for $G$ to have an approximate representation in some dimension $d$, but no exact representation of the same dimension: to obtain an example of this, take any group $G$ that has all non-trivial irreducible representations of large enough dimension, and create an approximate representation in e.g. dimension one less by ``cutting off'' one row and column from an exact representation. For sufficiently ``smooth'' $\sigma$ (no disproportionately large singular values) the dimension normalization induced by the norm $\|\cdot\|_\sigma$ will make this barely  noticeable, but it will be impossible to ``round'' the approximate representation obtained to an exact one without modifying the dimension. 

\begin{example}
\label{ex:wh-3}
Continuing with Example~\ref{ex:wh-1} we consider the example of $G= \mP$. In Example~\ref{ex:wh-1} we observed that a qubit $(\ket{\psi},X,Z)$ can be used to specify a $(0,\sigma)$ representation $f$ of $\mP$ such that moreover $f(-1)=-\Id$. We now check that the converse holds: for any $(0,\sigma)$-representation of $\mP$ for invertible $\sigma$, if $X=f(\sigma_X)$ and $Z=f(\sigma_Z)$ then using~\eref{eq:gh-condition-2}, taking $x=y=\sigma_X$ and $x=y=\sigma_Z$ it follows that $X^2 = Z^2=f(1)=\Id$, and taking $x=\sigma_X$ and $y=\sigma_Z$ we get that $XZ=f(\sigma_X \sigma_Z)$ while 
\[ZX=f(\sigma_Z\sigma_X)=f(-\sigma_X\sigma_Z)=f(-1)f(\sigma_X\sigma_Z)\;,\]
where the second equality uses that the Pauli anti-commute and the last equality again uses~\eref{eq:gh-condition-2}. Thus if $f(-1)=-\Id$ then 
 $\{X,Z\}=0$, so the $(0,\sigma)$-representation $f$ specifies a qubit and in particular Lemma~\ref{lem:qubit-2-rigid} on the structure of a qubit applies. As a result, we have shown that there exists a single  representation of $\mP$ such that $f(-1)=-\Id$, and that it is given by the Pauli matrices in dimension $2$. 
\footnote{The condition $f(-1)=-\Id$ is necessary, as there are four $1$-dimensional representations of $\mP$: all combinations $f(\sigma_X)=\pm 1$ and $f(\sigma_Z)=\pm 1$. We have found the right number of irreps: $1\cdot 2^2 + 4\cdot 1 = 8 = |\mP|$.}
\end{example}

\begin{exercise}
Using Exercise~\ref{ex:wh-2} and Example~\ref{ex:wh-3}, show that Theorem~\ref{thm:gh} for the case where $G=\mP$ implies Exercise~\ref{ex:approx-qubit}. 
\end{exercise}

\begin{exercise}
Show Theorem~\ref{thm:gh} for the case where $G=\mP$. \emph{[Hint: Adapt the proof ``by calculation''
of Proposition~\ref{prop:explicit-iso} to take into account $\eps$-approximations.]}
\end{exercise}

The main ingredient for the proof of Theorem~\ref{thm:gh} is an appropriate notion of Fourier transform over non-abelian groups. Given an irreducible representation $\rho: G \to U_{d_\rho}(\C)$, define 
\begin{equation}\label{eq:fourier}
 \hat{f}(\rho) \,=\, \Es{x\in G} \,f(x) \otimes \overline{\rho(x)}.
\end{equation}
In case $G$ is abelian, we always have $d_\rho=1$, the tensor product is a product, and \eref{eq:fourier} reduces to the usual definition of Fourier coefficient. The only properties we will need of irreducible representations is that they satisfy the relation
\begin{equation}
\label{eq:ortho}
\sum_\rho \,d_\rho\,\mathrm{Tr}(\rho(x)) \,=\, |G|\delta_{xe}\;,
\end{equation}
for any $x\in G$. Note that plugging in $x=e$ (the identity element in $G$) yields $\sum_\rho d_\rho^2= |G|$. 

\begin{proof}[Proof of Theorem \ref{thm:gh}]
Our first step is to define an isometry $V:\C^d \to \C^d \otimes (\oplus_\rho \C^{d_\rho} \otimes \C^{d_\rho})$ by
$$ V :\;u \in \C^d \,\mapsto\, \bigoplus_\rho \,d_\rho^{1/2} \sum_{i=1}^{d_\rho} \,\big(\hat{f}(\rho) (u\otimes e_i)\big) \otimes e_i,$$
where the direct sum ranges over all irreducible representations $\rho$ of $G$ and $\{e_i\}$ is the canonical basis.\footnote{Observe that this expression directly generalizes~\eref{eq:explicit-isometry}.}
Note what $V$ does: it ``embeds'' any vector $u\in \C^d$ into a direct sum, over irreducible representations $\rho$, of a $d$-dimensional vector of $d_\rho\times d_\rho$ matrices. Each (matrix) entry of this vector can be thought of as the Fourier coefficient of the corresponding entry of the vector $f(x)u$ associated with $\rho$. 
The fact that $V$ is an isometry follows from the appropriate extension of Parseval's formula:  
\begin{eqnarray*}
& V^* V &= \sum_\rho d_\rho \sum_i (I\otimes e_i^*) \hat{f}(\rho)^*\hat{f}(\rho) (I\otimes e_i)\\
&&= \Es{x,y}\,  f(x)^*f(y) \sum_\rho d_\rho \sum_i  (e_i^* \rho(x)^T \overline{\rho(y)} e_i)\\
&&= \sum_\rho \frac{d_\rho^2}{|G|}I = I,
\end{eqnarray*}
where for the second line we used the definition \eref{eq:fourier} of $\hat{f}(\rho)$ and  for the third we used \eref{eq:ortho} and the fact that $f$ takes values in the unitary group.

Next define
$$g(x) = \bigoplus_\rho \,\big(I_d \otimes I_{d_\rho} \otimes \rho(x)\big), $$
a direct sum over all irreducible representations of $G$ (hence itself a representation). Lets' first compute the ``pull-back'' of $g$ by $V$: following a similar calculation as above, for any $x\in G$, 
\begin{eqnarray*}
& V^*g(x) V  &=  \sum_{\rho}  d_\rho \sum_{i,j} (I\otimes e_i^*)\hat{f}(\rho)^* \hat{f}(\rho)(I\otimes e_j) \otimes e_i^* \rho(x) e_j ) \\
&& =  \Es{z,y}\,  f(z)^*f(y)  \sum_{\rho}  d_\rho \sum_{i,j} (e_i^* \rho(z)^T \overline{\rho(y)} e_j) \big( e_i^* \rho(x) e_j \big) \\
&& =  \Es{z,y}\,  f(z)^*f(y)  \sum_{\rho}  d_\rho \mathrm{Tr}\big( \rho(z)^T \overline{\rho(y)}  {\rho(x)^T} \big) \\
&& =  \Es{z,y}\,  f(z)^*f(y)  \sum_{\rho}  d_\rho \mathrm{Tr}\big( \rho(z^{-1}y x^{-1}) \big) \\
&& =  \Es{z}\,  f(z)^*f(zx) , 
\end{eqnarray*}
where the last equality uses \eref{eq:ortho}.
It then follows that 
\begin{eqnarray*}
&\Es{x}\, \big\langle f(x), V^*g(x) V \big\rangle_\sigma &=  \Es{x,z} \mathrm{Tr}\big( f(x) f(zx)^* f(z)\sigma\big).
\end{eqnarray*}  
This relates correlation of $f$ with $V^*gV$ to the quality of $f$ as an approximate representation and proves the theorem. 
\end{proof}


\section{Rigidity for the Magic Square game}
\label{section-msrigid}

Recall the Magic Square game from Section~\ref{section-msgame}. In Lemma~\ref{lemma-msperfect} we analyzed perfect strategies in this game and showed that any perfect strategy must ``have a qubit''. Remembering the proof, we had seen that Bob's observables in a perfect strategy must form an operator solution to the underlying system of equations, and that any such operator solution must contain two anti-commuting observables. Using Theorem~\ref{thm:gh} we can extend our earlier result to the case of approximate strategies with very little extra work. (This was first shown in~\cite{wu2016device} using a more direct proof.)

\begin{theorem}
\label{thm:rigid-ms}
Let $\eps>0$, and suppose that a strategy for the players  in the Magic Square game, using  a bipartite state $\ket{\psi}\in\C^d\otimes \C^d$ and observables $B_1,\ldots,B_9$ for Bob succeeds with probability $1-\eps$, for some $\eps \geq 0$. Then $(\ket{\psi},B_2,B_4)$ is an $O(\sqrt{\eps})$-approximate qubit. Moreover, there are local isometries $V_A,V_B:\C^d \to \C^2 \otimes \C^{d'}$ such that 
\begin{equation}
\label{eq:chsh-state}
\big\| V_A \otimes V_B \ket{\psi} - \ket{\phi^+} \otimes \ket{\psi'} \big\| = O(\sqrt{\eps}) \;,
\end{equation}
and 
\begin{equation}
\label{eq:chsh-alice}
\big\|( V_A \otimes V_B)( \Id \otimes B_2 )\ket{\psi} - (\sigma_Z \otimes \Id) \ket{\phi^+} \otimes \ket{\psi'} \big\| \,=\, O\big(\sqrt{\eps}\big)\;,
\end{equation}
and a similar relation holds with $B_2$ replaced by $B_4$ and $\sigma_Z$ replaced by $\sigma_X$. 
\end{theorem}

Note that this is a slightly weaker variant of Theorem~\ref{thm:ms-robust} that we stated without proof, because it only characterizes a single qubit of the strategy, instead of two for the case of Theorem~\ref{thm:ms-robust}. This will suffice for our purposes. 

\begin{proof}
For the first step of the proof we follow the proof of Claim~\ref{claim:ms-ac}, except that all equalities must be made approximate equalities. 
Assume without loss of generality that Alice's strategy is specified by six $4$-outcomes projective measurements $\{A^i_{a_1,a_2,a_3}\}$, where $a_1,a_2,a_3\in\{\pm 1\}$ range over the four possible assignments that satisfy the constraint associated with the $i$-th row $(i\in \{1,2,3\})$ or $(i-3)$-th column ($i\in \{4,5,6\}$). (We can assume that Alice always returns a valid assignment because she knows that she will lose if not, so we do not even need to include a symbol for such evidently wrong answers in the game.)

For any $i\in\{1,\ldots,9\}$, in addition to Bob's observable $B_i$ associated with question $i$ we define two observables from Alice's strategy, obtained by recording her answer associated to entry $i$ when she is asked the unique row containing $i$ --- call this observable $C_i$ --- or the unique column containing $i$ --- call this observable $C'_i$. Formally, $C_1 = \sum_{a_1,a_2,a_3\in\{\pm 1\}}a_1 A^1_{a_1,a_2,a_3}$, and similar relations can be used to define each $C_i$ and $C'_i$. Due to the parity constraints enforced on Alice's answers it is always the case that $C_1C_2C_3=+\Id,\ldots,C_3C_6C_9=-\Id$.
 
By definition, success of the strategy in the game implies $18$ equations of the form 
\begin{equation}
\label{eq:ms-bc-1}
 \bra{\psi} C_i \otimes B_i \ket{\psi} \,\geq \, 1-18\eps\qquad\text{and}\qquad \bra{\psi} C'_i \otimes B_i \ket{\psi} \,\geq \, 1-18\eps\;,
\end{equation}
for all $i\in \{1,\ldots,9\}$. Indeed, each such equation represents the probability that the players return valid answers, conditioned on each of the $18$ possible pairs of questions in the game. These relations allow us to mimic the proof of Claim~\ref{claim:ms-ac}, as follows: 
\begin{align*}
B_2 B_4 \ket{\psi} & \approx C_4 \otimes B_2 \ket{\psi}\\
&= C_5 C_6 \otimes B_2 \ket{\psi}\\
&\approx C_5 C_6 C_2 \otimes \Id \ket{\psi}\\
&= C_5 C_6 C_3 C_1 \otimes \Id \ket{\psi}\\
&\approx C_5C_6C_3 \otimes B_1 \ket{\psi}\\
&\approx C_5C_6\otimes B_1 B_3 \ket{\psi}\\
&\approx C_5C'_6\otimes B_1 B_3 \ket{\psi}\\
&\approx C_5C'_6 C'_3\otimes B_1  \ket{\psi}\\
&=  C_5C'_9\otimes B_1  \ket{\psi}\;,
\end{align*}
where here we use the notation $\ket{u}\approx \ket{v}$ to mean $\|\ket{u} - \ket{v}\|^2 = O(\eps)$. Here, each of the approximations is obtained by bounding the squared norm of the difference using the Cauchy-Scharz inequality and the required relation; for example, for the first approximation we write
\begin{align*}
 \big\|(\Id \otimes B_2 B_4 - C_4 \otimes B_2)\ket{\psi}\big\|^2
&= \big\|(\Id \otimes B_2)(\Id\otimes B_4 - C_4 \otimes \Id)\ket{\psi}\big\|^2\\
&= \big\|(\Id\otimes B_4 - C_4 \otimes \Id)\ket{\psi}\big\|^2\\ 
&= 2 - 2 \bra{\psi} B_4 \otimes C_4 \ket{\psi} \\
& \leq 36\eps\;,
\end{align*}
where the derivation uses that $B_2,B_4$ and $C_4$ are Hermitian and square to identity. 
Using a similar chain of approximations starting from $B_4 B_2 \ket{\psi}$, it follows that $(\ket{\psi},B_2,B_4)$ is an $O(\sqrt{\eps})$-approximate qubit.
 
As already observed in Exercise~\ref{ex:wh-2}, it follows that $B_2$ and $B_4$ induce an approximate representation of $\mP$ by setting 
$$ f(\pm \Id)=\pm\Id,\quad f(\pm\sigma_Z)=\pm B_2,\quad\quad f(\pm\sigma_X)=\pm B_4,\quad f(\pm\sigma_X\sigma_Z) = \pm B_4B_2\;.$$
Note that this is a legal definition, since $B_2$, $B_4$, and $B_2B_4$ are all unitary. Moreover, using only the approximate anti-commutation and the fact that $B_2$ and $B_4$ are observables it is immediate to verify that the conditions of Theorem~\ref{thm:gh} are satisfied, i.e. $f$ is an $(O({\eps}),\rho_\reg{B})$-representation of $\mP$, where $\rho_\reg{B}$ denotes the reduced density of $\ket{\psi}$ on $\mH_\reg{B}$.  

Applying the theorem, there must exist an exact representation $g$ of $\mP$ to which $f$ is close. However, as we saw in Example~\ref{ex:wh-3} the representation theory of $\mP$ is not complicated. It has four $1$-dimensional irreducible representations, but all of them map $-\Id$ to $1$, so they cannot be close to $f$. Since any representation is a direct sum of irreducible representations, and all irreducible representations of $\mP$ are far from $f$, $g$ must be a direct sum of multiple copies of the unique irreducible $2$-dimensional representation of $\mP$, which is precisely given by the Pauli matrices, together with possibly a small (relative to the total dimension) number of $1$-dimensional relations. Ignoring the presence of such representations for simplicity,\footnote{To account for them we would select the ``corner'' of the range of the isometry $V$ that includes only copies of the $2$-dimensional irrep; this is a simple technicality.} $g(\sigma_Z) = \sigma_Z \otimes \Id$ and $g(\sigma_X) = \sigma_X \otimes \Id$, which gives~\eref{eq:chsh-alice}. 

To conclude~\eref{eq:chsh-state} we observe that the relations~\eref{eq:ms-bc-1} imply that also $(\rho_\reg{A},C_2,C_4)$ is an $O(\sqrt{\eps})$-approximate qubit. This allows us to define an approximate representation of $\mP$ on $\mH_\reg{A}$, and apply Theorem~\ref{thm:gh} again to obtain on isometry $V_\reg{A}$ which maps $(\rho_\reg{A},C_2,C_4)$ close to an exact qubit. Letting 
\[ \ket{\psi'} = V_\reg{A} \otimes V_\reg{B} \ket{\psi} \in (\C^2 \otimes \mH_\reg{A}')\otimes (\C^2 \otimes \mH_\reg{B}') \]
we see from~\eref{eq:ms-bc-1} and~\eref{eq:chsh-alice} that 
\[\big| \bra{\psi} \big(\sigma_X \otimes \sigma_X \otimes \Id_{\reg{A}'\reg{B}'} + \sigma_Z \otimes \sigma_Z \otimes \Id_{\reg{A}'\reg{B}'} \big) \ket{\psi}\big| \geq 1-O({\eps})\;.\]
To conclude note that $\frac{1}{2}(\sigma_X \otimes \sigma_X + \sigma_Z \otimes \sigma_Z )$ is an observable with a single eigenvalue $1$, with associated eigenvector $\ket{\phi^+}$, and all other eigenvalues equal to $0$ or $-1$. 
\end{proof}


\section{Chapter notes}
\label{section-rigidity-notes}


\begin{multicols}{2}[\section{Chapters}]
\noindent
Preliminaries
\begin{enumerate}
\item \hyperref[introduction-section-phantom]{Introduction}
\item \hyperref[notation-section-phantom]{Notation}
\end{enumerate}
Background
\begin{enumerate}
\setcounter{enumi}{2}
\item \hyperref[nonlocalgames-section-phantom]{Nonlocal games}
\item \hyperref[complexitytheory-section-phantom]{Complexity theory}
\item \hyperref[operatoralgebras-section-phantom]{Operator algebras}
\end{enumerate}
Warm-up
\begin{enumerate}
\setcounter{enumi}{5}
\item \hyperref[rigidity-section-phantom]{Rigidity}
\item \hyperref[blsgames-section-phantom]{Binary Linear System games}
\item \hyperref[paulibraiding-section-phantom]{Pauli braiding}
\end{enumerate}
Overview
\begin{enumerate}
\setcounter{enumi}{8}
\item \hyperref[argument-section-phantom]{The argument}
\item \hyperref[questionreduction-section-phantom]{Question reduction}
\item \hyperref[answerreduction-section-phantom]{Answer reduction}
\item \hyperref[recursivecompression-section-phantom]{Recursive compression}
\end{enumerate}
Building blocks
\begin{enumerate}
\setcounter{enumi}{12}
\item \hyperref[classicalldt-section-phantom]{Classical low-degree test}
\item \hyperref[quantumldt-section-phantom]{Quantum low-degree test}
\end{enumerate}
Related tools
\begin{enumerate}
\setcounter{enumi}{14}
\item \hyperref[parallelrepetition-section-phantom]{Parallel repetition}
\end{enumerate}
Extensions
\begin{enumerate}
\setcounter{enumi}{15}
\item TBD
\end{enumerate}
\end{multicols}

\bibliography{my}
\bibliographystyle{amsalpha}

\end{document}