%\documentclass{stacks-project-book}
\documentclass{amsart}
\usepackage{amsmath}

% For dealing with references we use the comment environment
\usepackage{verbatim}
\newenvironment{reference}{\comment}{\endcomment}
\newenvironment{slogan}{\comment}{\endcomment}
\newenvironment{history}{\comment}{\endcomment}


% We use multicol for the list of chapters between chapters
\usepackage{multicol}

% This is generall recommended for better output
\usepackage{lmodern}
\usepackage[T1]{fontenc}

% For cross-file-references
\usepackage{xr-hyper}

% Package for hypertext links:
\usepackage{hyperref}

% For any local file, say "hello.tex" you want to link to please
% use \externaldocument[hello-]{hello}
\externaldocument[introduction-]{introduction}
\externaldocument[-]{}
\externaldocument[nonlocalgames-]{}
\externaldocument[complexitytheory-]{}
\externaldocument[operatoralgebras-]{}

\externaldocument[paulibraiding-]{paulibraiding}


% Theorem environments.
%
\theoremstyle{plain}
\newtheorem{theorem}[subsection]{Theorem}
\newtheorem{proposition}[subsection]{Proposition}
\newtheorem{lemma}[subsection]{Lemma}
\newtheorem{fact}[subsection]{Fact}


\theoremstyle{definition}
\newtheorem{definition}[subsection]{Definition}
\newtheorem{example}[subsection]{Example}
\newtheorem{exercise}[subsection]{Exercise}
\newtheorem{situation}[subsection]{Situation}

\theoremstyle{remark}
\newtheorem{remark}[subsection]{Remark}
\newtheorem{remarks}[subsection]{Remarks}

\numberwithin{equation}{subsection}


%%%%%%%%%%
%% Macros

% Notation

\newcommand{\cal}[1]{\mathcal{#1}}
\newcommand{\mH}{\mathcal{H}}
\newcommand{\eps}{\varepsilon}
\newcommand{\mA}{\mathcal{A}}
\newcommand{\mX}{\mathcal{X}}
\newcommand{\mY}{\mathcal{Y}}
\newcommand{\mR}{\mathcal{R}}
\newcommand{\mG}{\mathcal{G}}


% Complexity classes
\newcommand{\NP}{\textsc{NP}}

% Probability

\newcommand{\Es}[1]{\textsc{E}_{#1}}
\newcommand{\E}{\mathop{\mathbb{E}}\displaylimits} % Expectation


% Math

\newcommand{\C}{\mathbb{C}}
\newcommand{\N}{\mathbb{N}}

\newcommand{\Tr}{\mbox{\rm Tr}}
\newcommand{\Id}{\ensuremath{\mathop{\rm Id}\nolimits}}

\newcommand{\norm}[1]{\left\| {#1} \right\|}

\newcommand{\eval}{\mathrm{eval}}
\newcommand{\poly}{\mathrm{poly}}



% Quantum notation

\newcommand{\ket}[1]{|#1\rangle}
\newcommand{\bra}[1]{\langle#1|}
\newcommand{\ketbra}[2]{\ket{#1}\!\bra{#2}}
\newcommand{\proj}[1]{\ket{#1}\!\bra{#1}}

% Games

\newcommand{\game}{\mathfrak{G}}
\newcommand{\strategy}{{S}}
\newcommand{\val}{\ensuremath{\mathrm{val}}}
\newcommand{\Ent}{\mathcal{E}}
\newcommand{\alice}{A}
\newcommand{\bob}{B}

\newcommand{\abc}[1][\delta]{\otimes I_\bob \simeq_{#1} I_\alice \otimes}

\begin{document}

\title{Notation}
\label{notation}

\maketitle

\phantomsection
\label{section-phantom}

\tableofcontents

We summarize notation used throughout. Notation local to a chapter is introduced in the relevant chapter. 

\section{Mathematical notation}
\label{section-mathnotation}

We use $\mH$ to denote a separable Hilbert space. Unless explicitly specified otherwise, $\mH$ is assumed to be finite dimensional. We use $\Id_\mH$ to denote the identity operator on $\mH$, sometimes abbreviated as $\Id$. We use $\Tr(\cdot)$ to denote the standard trace on $\mH$. 

\section{Quantum information}
\label{section-qinotation}

We use standard notation as used in e.g. the book~\cite{nielsen2000quantum}. 

A (pure) quantum state $\ket{\psi}$ is a unit vector in $\mH$. A density matrix on $\mH$ is a positive semidefinite operator $\rho$ on $\mH$ such that $\Tr(\rho)=1$. 

\begin{definition}[POVM]
\label{definition-povm}
A positive operator-valued measure (POVM) on $\mH$ is a finite collection $\{A_i\}$ of positive semidefinite operators on $\mH$ such that $\sum_i A_i = \Id$. 
\end{definition}

A POVM $\{A_i\}$ is sometimes referred to as a \emph{measurement}. If all $A_i$ are projections then the measurement is called \emph{projective}. 



\begin{multicols}{2}[\section{Chapters}]
\noindent
Preliminaries
\begin{enumerate}
\item \hyperref[introduction-section-phantom]{Introduction}
\item \hyperref[notation-section-phantom]{Notation}
\end{enumerate}
Background
\begin{enumerate}
\setcounter{enumi}{2}
\item \hyperref[nonlocalgames-section-phantom]{Nonlocal games}
\item \hyperref[complexitytheory-section-phantom]{Complexity theory}
\item \hyperref[operatoralgebras-section-phantom]{Operator algebras}
\end{enumerate}
Warm-up
\begin{enumerate}
\setcounter{enumi}{5}
\item \hyperref[rigidity-section-phantom]{Rigidity}
\item \hyperref[blsgames-section-phantom]{Binary Linear System games}
\item \hyperref[paulibraiding-section-phantom]{Pauli braiding}
\end{enumerate}
Overview
\begin{enumerate}
\setcounter{enumi}{8}
\item \hyperref[argument-section-phantom]{The argument}
\item \hyperref[questionreduction-section-phantom]{Question reduction}
\item \hyperref[answerreduction-section-phantom]{Answer reduction}
\item \hyperref[recursivecompression-section-phantom]{Recursive compression}
\end{enumerate}
Building blocks
\begin{enumerate}
\setcounter{enumi}{12}
\item \hyperref[classicalldt-section-phantom]{Classical low-degree test}
\item \hyperref[quantumldt-section-phantom]{Quantum low-degree test}
\end{enumerate}
Related tools
\begin{enumerate}
\setcounter{enumi}{14}
\item \hyperref[parallelrepetition-section-phantom]{Parallel repetition}
\end{enumerate}
Extensions
\begin{enumerate}
\setcounter{enumi}{15}
\item TBD
\end{enumerate}
\end{multicols}


\bibliography{my}
\bibliographystyle{amsalpha}

\end{document}