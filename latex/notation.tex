
\documentclass{stacks-project-book}

% For dealing with references we use the comment environment
\usepackage{verbatim}
\newenvironment{reference}{\comment}{\endcomment}
%\newenvironment{reference}{}{}
\newenvironment{slogan}{\comment}{\endcomment}
\newenvironment{history}{\comment}{\endcomment}


% We use multicol for the list of chapters between chapters
\usepackage{multicol}

% This is generall recommended for better output
\usepackage{lmodern}
\usepackage[T1]{fontenc}

% For cross-file-references
\usepackage{xr-hyper}

% Package for hypertext links:
\usepackage{hyperref}

% For any local file, say "hello.tex" you want to link to please
% use \externaldocument[hello-]{hello}
\externaldocument[introduction-]{introduction}
\externaldocument[paulibraiding-]{paulibraiding}


% Theorem environments.
%
\theoremstyle{plain}
\newtheorem{theorem}[subsection]{Theorem}
\newtheorem{proposition}[subsection]{Proposition}
\newtheorem{lemma}[subsection]{Lemma}

\theoremstyle{definition}
\newtheorem{definition}[subsection]{Definition}
\newtheorem{example}[subsection]{Example}
\newtheorem{exercise}[subsection]{Exercise}
\newtheorem{situation}[subsection]{Situation}

\theoremstyle{remark}
\newtheorem{remark}[subsection]{Remark}
\newtheorem{remarks}[subsection]{Remarks}

\numberwithin{equation}{subsection}


%%%%%%%%%%
%% Macros


% Complexity classes
\newcommand{\NP}{\textsc{NP}}

% Probability

\newcommand{\Es}[1]{\textsc{E}_{#1}}

% Linear algebra

\newcommand{\Tr}{\mbox{\rm Tr}}
\newcommand{\Id}{\ensuremath{\mathop{\rm Id}\nolimits}}

% Quantum notation

\newcommand{\ket}[1]{|#1\rangle}
\newcommand{\bra}[1]{\langle#1|}
\newcommand{\proj}[1]{\ket{#1}\!\bra{#1}}



% OK, start here.
%
\begin{document}


\title{Notation}
\label{notation}

\maketitle

\phantomsection
\label{section-phantom}

\tableofcontents

We summarize notation used throughout. Notation local to a chapter is introduced in the relevant chapter. 


\section{Mathematical notation}
\label{section-mathnotation}

We use $\mH$ to denote a separable Hilbert space. Unless explicitly specified otherwise, $\mH$ is assumed to be finite dimensional. We use $\Id_\mH$ to denote the identity operator on $\mH$, sometimes abbreviated as $\Id$. We use $\Tr(\cdot)$ to denote the standard trace on $\mH$. 

\section{Quantum information}
\label{section-qinotation}

We use standard notation as used in e.g. the book~\cite{nielsen2000quantum}. 

A (pure) quantum state $\ket{\psi}$ is a unit vector in $\mH$. A density matrix on $\mH$ is a positive semidefinite operator $\rho$ on $\mH$ such that $\Tr(\rho)=1$. 

\begin{definition}[POVM]
\label{definition-povm}
A positive operator-valued measure (POVM) on $\mH$ is a finite collection $\{A_i\}$ of positive semidefinite operators on $\mH$ such that $\sum_i A_i = \Id$. 
\end{definition}

A POVM $\{A_i\}$ is sometimes referred to as a \emph{measurement}. If all $A_i$ are projections then the measurement is called \emph{projective}. 



\begin{multicols}{2}[\section{Chapters}]
\noindent
Preliminaries
\begin{enumerate}
\item \hyperref[introduction-section-phantom]{Introduction}
\end{enumerate}
Self-testing
\begin{enumerate}
\setcounter{enumi}{1}
\item \hyperref[paulibraiding-section-phantom]{Pauli Braiding}
\end{enumerate}
Complexity
\begin{enumerate}
\setcounter{enumi}{1}
\item TBD
\end{enumerate}
\end{multicols}


\bibliography{my}
\bibliographystyle{amsalpha}

\end{document}